\documentclass{beamer}
\usepackage{graphicx}
\usepackage{amsmath}
\usepackage{xcolor}

\usetheme{Madrid}

\title{Week 11: Hypothesis Test for Median and Variance}
\author{Kaosar}
\institute{Auburn University}
\date{Spring 2025}

\begin{document}

\begin{frame}
\titlepage
\end{frame}

\begin{frame}{Administrative Items}
\begin{itemize}
    \item Tuesday, April 10: second restricted resource lab
    \item R help
    \item Note sheet updated in each lab
\end{itemize}
\end{frame}

\begin{frame}[fragile]{Lab Instructions: Step 1}
Tests for the median using the islands dataset:
\begin{verbatim}
install.packages("ggpubr")
library(ggpubr)
data("islands")

qqnorm(islands)
qqline(islands)

sample1 <- sample(islands, 15)
# Repeat for sample2 to sample5
\end{verbatim}
\end{frame}

\begin{frame}[fragile]{Lab Instructions: Boxplot Example}
\begin{verbatim}
sampleIndices <- rep(1:5, each = 15)
allSampleData <- c(sample1, sample2, sample3, sample4, sample5)

my_data <- data.frame(sampleIndices, allSampleData)
colnames(my_data) <- c("sampleNumber", "Landmass")

library(ggpubr)
ggboxplot(my_data, x="sampleNumber", y="Landmass")
\end{verbatim}
\end{frame}

\begin{frame}{Sign Test }
\textbf{Sign Test:}
\begin{itemize}
    \item Nonparametric test used to determine if the median of a population is significantly different from a hypothesized median.
    \item Counts the number of observations above and below the hypothesized median.
    \item Does not account for magnitude, only the direction (sign).
\end{itemize}
\end{frame}
\begin{frame}{ Wilcoxon Signed-Rank Test}
\textbf{Wilcoxon Signed-Rank Test:}
\begin{itemize}
    \item Nonparametric test that considers both the direction and magnitude of differences.
    \item Ranks the absolute differences between each observation and the hypothesized median, then sums ranks separately for positive and negative differences.
    \item Provides more power compared to the sign test due to consideration of ranks.
\end{itemize}
\end{frame}
\begin{frame}[fragile]{Hypothesis Testing for Median}
\begin{verbatim}
# Sign test
binom.test(sample1, median=100)

# Signed-rank test
wilcox.test(sample1, mu=100)
\end{verbatim}
Repeat for all samples.
\end{frame}

\begin{frame}{Median and Variance Tests}
\textbf{Median Test:}
\begin{itemize}
    \item Used to test if two or more samples come from populations with the same median.
    \item Based on counting observations above and below the overall median.
\end{itemize}

\textbf{Variance Test (F-test):}
\begin{itemize}
    \item Compares the variances of two populations.
    \item Test statistic:
\end{itemize}
\[
F = \frac{s_1^2}{s_2^2}
\]
where \( s_1^2 \) and \( s_2^2 \) are the sample variances.
\end{frame}
\begin{frame}[fragile]{Tests for Median and Variance: Iris Dataset}
\begin{verbatim}
data("iris")
library(dplyr)
iris_data <- iris %>% select(Sepal.Length, Species)

setosa <- iris_data %>% filter(Species == "setosa")
versicolor <- iris_data %>% filter(Species == "versicolor")

hist(setosa$Sepal.Length)
qqnorm(setosa$Sepal.Length)
qqline(setosa$Sepal.Length)
\end{verbatim}
\end{frame}

\begin{frame}[fragile]{Hypothesis Testing in Iris Dataset}
\begin{verbatim}
# Median test
wilcox.test(setosa$Sepal.Length, versicolor$Sepal.Length)

# Variance test
var.test(setosa$Sepal.Length, versicolor$Sepal.Length)
\end{verbatim}
\end{frame}

\begin{frame}{Linear Regression: Theory}
Linear regression describes the relationship:
\[
Y = \beta_0 + \beta_1 X + \epsilon
\]
\begin{itemize}
    \item $\beta_0$: intercept
    \item $\beta_1$: slope
    \item $\epsilon$: error term
\end{itemize}
\end{frame}

\begin{frame}{Linear Regression: Least Squares Method}
Coefficients estimated by:
\[
\hat{\beta}_1 = \frac{\sum(x_i - \bar{x})(y_i - \bar{y})}{\sum(x_i - \bar{x})^2}, \quad \hat{\beta}_0 = \bar{y} - \hat{\beta}_1\bar{x}
\]
\end{frame}

\begin{frame}[fragile]{Linear Regression Example in R}
\begin{verbatim}
data(cars)
model <- lm(dist ~ speed, data=cars)
plot(cars$speed, cars$dist)
abline(model, col="blue")
summary(model)
\end{verbatim}
\end{frame}

\begin{frame}{Lab Submissions}
\begin{itemize}
    \item Submit updated note sheet
    \item Clearly commented R script
    \item Ensure script runs without errors
    \item Report using Canvas template
    \item Include all plots and summaries
\end{itemize}
\end{frame}

\end{document}
% trigger AI polish
