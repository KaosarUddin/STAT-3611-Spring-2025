\documentclass{article}
\usepackage{listings}
\usepackage{xcolor}

\lstset{
  language=R,
  basicstyle=\ttfamily\small,
  keywordstyle=\color{blue},
  commentstyle=\color{green!60!black},
  stringstyle=\color{red},
  frame=single,
  breaklines=true,
  numbers=left,
  numberstyle=\tiny,
  captionpos=b
}

\begin{document}

\section*{STAT 3611 Lab Test 2 Solution}

\subsection*{1. Mice Dataset Overview and Difference Analysis}

\begin{lstlisting}[language=R]
library(datarium)
data("mice2", package = "datarium")
nrow(mice2)

mean_before <- mean(mice2$before)
sd_before <- sd(mice2$before)
mean_after <- mean(mice2$after)
sd_after <- sd(mice2$after)

mice2$diff <- mice2$after - mice2$before
mean_diff <- mean(mice2$diff)
var_diff <- var(mice2$diff)

hist(mice2$diff, main = "Histogram of Weight Differences", col = "lightblue")
qqnorm(mice2$diff, main = "QQ plot of Weight Differences")
qqline(mice2$diff, col = "red")
\end{lstlisting}

\textbf{Answer:} The dataset contains 10 rows. The histogram and QQ plot suggest the differences are approximately normally distributed.

\subsection*{2. Paired t-tests on Mice Dataset}

\begin{lstlisting}[language=R]
# Two-sided paired t-test
t.test(mice2$after, mice2$before, paired = TRUE, conf.level = 0.98)

# One-sided paired t-test (greater than threshold)
threshold <- 0.5
t.test(mice2$diff, mu = threshold, alternative = "greater", conf.level = 0.98)
\end{lstlisting}

\textbf{Answer:}  
- Null: No mean difference between before and after weights.  
- P-value from two-sided test was extremely small, so we reject the null hypothesis.  
- For one-sided test, again we reject the null; weight after treatment is significantly greater.

\subsection*{3. Expenses Dataset Overview}

\begin{lstlisting}[language=R]
expenses <- read.csv("expenses.csv")
dim(expenses)
colnames(expenses)
unique_categories <- unique(expenses$region)
length(unique_categories)
\end{lstlisting}

\textbf{Answer:} The dataset has 1,338 rows and 7 columns. There are 4 regions: southeast, southwest, northwest, northeast.

\subsection*{4. Region-Based Summaries and Boxplots}

\begin{lstlisting}[language=R]
category_dfs <- split(expenses, expenses$region)
sapply(category_dfs, nrow)
lapply(category_dfs, function(df) summary(df$charges))

boxplot(charges ~ region, data = expenses, ylim = c(0, 50000),
        main = "Charges by Region",
        xlab = "Region", ylab = "Charges ($)", col = rainbow(4))
\end{lstlisting}

\textbf{Answer:} Boxplots and summaries show clear differences in charges by region, with southeast generally having higher charges.

\subsection*{5. Confidence Intervals and Hypothesis Test for Charges}

\begin{lstlisting}[language=R]
charges_se <- expenses$charges[expenses$region == "southeast"]
charges_nw <- expenses$charges[expenses$region == "northwest"]

# 99% Confidence Intervals
t.test(charges_se, conf.level = 0.99)$conf.int
t.test(charges_nw, conf.level = 0.99)$conf.int

# Hypothesis Test
t.test(charges_se, charges_nw, alternative = "two.sided", conf.level = 0.95)
\end{lstlisting}

\textbf{Answer:}  
- $H_0$: $\mu_{se} = \mu_{nw}$, $H_a$: $\mu_{se} \ne \mu_{nw}$.  
- We reject $H_0$ based on a small p-value, concluding the average charges differ between southeast and northwest.

\subsection*{6. Linear Regression: Charges vs Age}

\begin{lstlisting}[language=R]
lm_model <- lm(charges ~ age, data = expenses)
plot(expenses$age, expenses$charges, main = "Charges vs Age",
     xlab = "Age", ylab = "Charges ($)", pch = 19, col = "blue")
abline(lm_model, col = "red")
summary(lm_model)
\end{lstlisting}

\textbf{Answer:}  
- Regression equation: $\hat{y} = \beta_0 + \beta_1 x$ where $\beta_1$ (slope) is positive.  
- Slope indicates that charges tend to increase with age.  
- $R^2$ indicates moderate explanatory power.

\subsection*{7. Linear Regression: Mice Dataset}

\begin{lstlisting}[language=R]
mice_model <- lm(after ~ before, data = mice2)
plot(mice2$before, mice2$after, main = "Mouse Weights Before vs After Treatment",
     xlab = "Before Weight", ylab = "After Weight", pch = 19, col = "purple")
abline(mice_model, col = "green")
summary(mice_model)
\end{lstlisting}

\textbf{Answer:}  
- The regression slope indicates a strong positive linear relationship between initial and final weight.  
- The $R^2$ value confirms a good model fit.

\end{document}
