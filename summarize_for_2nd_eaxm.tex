\documentclass{article}
\usepackage{graphicx}
\usepackage{amsmath}
\usepackage{hyperref}
\usepackage{listings}
\usepackage{xcolor}

\lstset{
  language=R,
  basicstyle=\ttfamily\footnotesize,
  keywordstyle=\color{blue},
  commentstyle=\color{green},
  stringstyle=\color{red},
  numbers=left,
  numberstyle=\tiny,
  stepnumber=1,
  breaklines=true,
  showstringspaces=false
}

\begin{document}

\title{Review for Test 2}
\author{Kaosar, Auburn University}
\date{Spring 2025}
\maketitle
This document provides a concise summary of key concepts and skills covered across Labs 6 to 11 in STAT 3611. It emphasizes important statistical procedures, data handling techniques, and R programming tools.

\section*{Lab 6: Confidence Intervals for the Mean}
\begin{itemize}
  \item Use of built-in R dataset \texttt{faithful}.
  \item Compute population and sample statistics: mean, variance, standard deviation, coefficient of variation.
  \item Sampling techniques using \texttt{sample()} and \texttt{replicate()}.
  \item Confidence intervals for the mean using \texttt{t.test()}.
  \item Bias calculation for sample estimates.
  \item Data export using \texttt{write.csv()}.
\end{itemize}

\section*{Lab 7: Confidence Intervals for Standard Deviation and Proportion}
\begin{itemize}
  \item Add computation of sample proportions (\texttt{p-hat}).
  \item Construct CI for standard deviation using chi-squared distribution and \texttt{qchisq()}.
  \item Construct CI for proportions using normal approximation.
  \item Boxplot visualization of variance and \texttt{p-hat} across samples.
\end{itemize}

\section*{Lab 9: Hypothesis Testing (Mean)}
\begin{itemize}
  \item Test known mean with both known and unknown variance.
  \item Two-sided z-test and t-test using \texttt{qnorm()}, \texttt{t.test()}.
  \item Type II error calculation with \texttt{pnorm()}.
  \item One-sided t-test example using data subset (AirPassengers).
\end{itemize}

\section*{Lab 10: Hypothesis Testing for Mean and Proportion}
\begin{itemize}
  \item Use of the \texttt{birthwt} and \texttt{precip} datasets.
  \item Add new variables (e.g., converting weight from grams to pounds).
  \item Two-sample t-tests for comparing group means.
  \item Hypothesis testing for proportions using \texttt{binom.test()} and \texttt{prop.test()}.
  \item QQ plot and skewness with \texttt{e1071} library.
\end{itemize}

\section*{Lab 11: Median and Variance Testing}
\begin{itemize}
  \item Tests for medians using \texttt{binom.test()} and \texttt{wilcox.test()}.
  \item Use of \texttt{ggboxplot()} from \texttt{ggpubr} package for grouped boxplots.
  \item Variance test using \texttt{var.test()}.
  \item Linear regression with \texttt{lm()} and plot with \texttt{abline()}.
  \item R-squared interpretation and slope inference.
\end{itemize}

\section*{Common R Functions and Tips}
\begin{itemize}
  \item \texttt{mean()}, \texttt{var()}, \texttt{sd()}, \texttt{qnorm()}, \texttt{qt()}, \texttt{qchisq()}, \texttt{sample()}, \texttt{replicate()}
  \item \texttt{t.test()}, \texttt{binom.test()}, \texttt{prop.test()}, \texttt{wilcox.test()}, \texttt{var.test()}, \texttt{lm()}
  \item Use \texttt{par(mfrow=c(...))} to layout multiple plots.
  \item Always clean missing values using \texttt{na.omit()} when needed.
  \item Export data with \texttt{write.csv()}.
\end{itemize}

\end{document}
