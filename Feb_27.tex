\documentclass{beamer}
\usepackage{graphicx}
\usepackage{amsmath}
\usepackage{xcolor}
\usepackage{tcolorbox}  % For colored code blocks
\usepackage{verbatim}   % For raw code display

\usetheme{Madrid}

\title{Week 6: Confidence Intervals for the Mean}
\author{Kaosar}
\institute{Auburn University}
\date{Spring 2025}

% Define code box style
\tcbuselibrary{listingsutf8}
\tcbset{
    colframe=black,
    colback=white,
    coltitle=black,
    boxrule=0.5mm,
    arc=2mm,
    fonttitle=\bfseries,
    left=5pt,
    right=5pt,
    top=5pt,
    bottom=5pt
}

\begin{document}

\frame{\titlepage}

\begin{frame}{Admin Stuff}
    \begin{itemize}
        \item Previous lab: closed-resources lab
        \item Note Sheet reminder
        \item One closed-resources lab remaining
    \end{itemize}
\end{frame}

% Steps 1-2: Load and Analyze Data
\begin{frame}{Step 1-2: Load and Analyze Data}
    \textbf{Load dataset and compute statistics}
    \begin{tcolorbox}[title=R Code]
\begin{verbatim}
# Load dataset
library(datasets)
data(iris)  # Load the iris dataset
t <- iris  # Assign dataset to variable

# Identify column headers
colnames(t)

# Number of rows
nrow(t)

# Compute statistics for numerical columns
summary_stats <- data.frame(
  Mean = sapply(t, function(x) if(is.numeric(x)) mean(x, na.rm = TRUE) else NA),
  Variance = sapply(t, function(x) if(is.numeric(x)) var(x, na.rm = TRUE) else NA),
  Std_Dev = sapply(t, function(x) if(is.numeric(x)) sd(x, na.rm = TRUE) else NA)
)

# Display statistics
print(summary_stats)
\end{verbatim}
    \end{tcolorbox}
\end{frame}

% Step 3: Sampling
\begin{frame}{Step 3: Sampling}
    \textbf{Generating Samples}
    \begin{tcolorbox}[title=R Code]
\begin{verbatim}
# Generate a random sample
x <- 1:100  # Define population
n <- 10  # Define sample size
sampled_values <- sample(x, n)  # Generate a random sample
print(sampled_values)

# Generate multiple samples
m <- 5  # Number of samples
sampled_data <- data.frame(replicate(m, sample(x, n)))
print(sampled_data)
\end{verbatim}
    \end{tcolorbox}
\end{frame}

% Step 4: Analyze Samples
\begin{frame}{Step 4: Analyze the Samples}
    \textbf{Compute Sample Statistics}
    \begin{tcolorbox}[title=R Code]
\begin{verbatim}
# Compute sample statistics
samples <- replicate(m, sample(x, n), simplify = FALSE)
sample_means <- c()
sample_variances <- c()
sample_std_devs <- c()

for (i in 1:m) {
  sample_means <- c(sample_means, mean(samples[[i]]))
  sample_variances <- c(sample_variances, var(samples[[i]]))
  sample_std_devs <- c(sample_std_devs, sd(samples[[i]]))
}

# Store results in a data frame
results <- data.frame(
  Sample_ID = 1:m,
  Mean = sample_means,
  Variance = sample_variances,
  Std_Dev = sample_std_devs
)
print(results)
\end{verbatim}
    \end{tcolorbox}
\end{frame}

% Step 6: Confidence Intervals
\begin{frame}{Step 6: Confidence Intervals}
    \textbf{Computing Confidence Intervals}
    \begin{tcolorbox}[title=R Code]
\begin{verbatim}
# Compute confidence intervals
set.seed(123)
true_mean <- 50
samples <- replicate(m, sample(x, n), simplify = FALSE)
sample_sds <- c()
lower_bounds <- c()
upper_bounds <- c()
contains_true_mean <- c()

for (i in 1:m) {
  sample_sds <- c(sample_sds, sd(samples[[i]]))
  margin_of_error <- t_value * (sample_sds[i] / sqrt(n))
  lower_bound <- sample_means[i] - margin_of_error
  upper_bound <- sample_means[i] + margin_of_error
  lower_bounds <- c(lower_bounds, lower_bound)
  upper_bounds <- c(upper_bounds, upper_bound)
  contains_true_mean <- c(contains_true_mean, (true_mean >= lower_bound & true_mean <= upper_bound))
}

# Store confidence interval results in a data frame
results <- data.frame(
  Sample_ID = 1:m,
  Sample_Mean = sample_means,
  Sample_SD = sample_sds,
  Lower_Bound = lower_bounds,
  Upper_Bound = upper_bounds,
  Contains_True_Mean = contains_true_mean
)
print(results)
\end{verbatim}
    \end{tcolorbox}
\end{frame}

% Step 8: Write to CSV
\begin{frame}{Step 8: Write to CSV}
    \textbf{Save Results to File}
    \begin{tcolorbox}[title=R Code]
\begin{verbatim}
# Write the data frame to a CSV file
output_file <- "C:/Users/ksrru/Documents/STAT_3611/samples_results.csv"
write.csv(samples_df, file = output_file, row.names = TRUE)
\end{verbatim}
    \end{tcolorbox}
\end{frame}

\end{document}

