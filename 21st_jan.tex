\documentclass{beamer}
\usepackage{graphicx}
\usepackage{amsmath}

\usetheme{Madrid}

\title{Week 2 : Data Description in R}
\author{Kaosar}
\institute{Auburn University}
\date{Spring 2025}

\begin{document}

\frame{\titlepage}

\begin{frame}{R Indexing: Zero vs One-Based}
\textbf{Indexing in Programming Languages:}
\begin{itemize}
    \item Many programming languages use \textbf{zero-based indexing}, where the first element of a vector is indexed as 0.
    \item R uses \textbf{one-based indexing}, where the first element of a vector is indexed as 1.
\end{itemize}
\end{frame}

\begin{frame}[fragile]{R Help and Admin Notes}
\textbf{Accessing Help in R:}
\begin{itemize}
    \item Use \texttt{help(functionName)} or \texttt{?functionName} for help:
    \begin{verbatim}
help(hist)    # Help for hist()
?mean         # Help for mean()
    \end{verbatim}
    \item Help appears on the right side of the RStudio screen.
\end{itemize}
\end{frame}

\begin{frame}[fragile]{Using R Scripts and Datasets}
\textbf{R Scripts:}
\begin{itemize}
    \item Save and edit code for future use.
    \item Use the script window in RStudio to write and execute scripts.
\end{itemize}

\textbf{Built-in Datasets:}
\begin{itemize}
    \item R includes datasets like \texttt{iris}.
    \item Load a dataset using the \texttt{data()} function:
\begin{verbatim}
data(iris)
\end{verbatim}
\end{itemize}
\end{frame}

\begin{frame}[fragile]{Dataset Structure}
\textbf{Exploring Dataset Structure:}
\begin{itemize}
    \item \texttt{dim(dataSetName)}: Number of rows and columns.
    \item \texttt{nrow()} and \texttt{ncol()}: Rows and columns separately.
    \item \texttt{head(dataSetName)}: First 6 rows (default).
\end{itemize}
\end{frame}



\begin{frame}[fragile]{Working with Columns}
\begin{itemize}
    \item Identify how many columns of data there are:
    \begin{verbatim}
length(dataSetName)
length(iris)
    \end{verbatim}

    \item Access one column:
    \begin{verbatim}
iris$Species
iris[5]
    \end{verbatim}

    \item Access multiple columns:
    \begin{verbatim}
# By column index range
dataSetName[c(colIndex3:colIndex4)]
iris[c(3:4)]

# By column names
subset(dataSetName, select = c(colName3, colName4))
subset(iris, select = c(Petal.Length, Petal.Width))
    \end{verbatim}
\end{itemize}
\end{frame}

\begin{frame}[fragile]{Working with Rows}
\textbf{Subsetting Rows:}
\begin{itemize}
    \item Use \texttt{subset()} with conditions to filter rows:
    \begin{verbatim}
subset(iris, iris$Petal.Width >= 2.2)
    \end{verbatim}

    \item Direct row criteria with square brackets:
    \begin{verbatim}
dataName[rowCriteria, columnIndex]
iris[iris$Petal.Width >= 2.2, ]
    \end{verbatim}

    \item \textbf{Important:} Always include a comma, even when selecting all columns.
\end{itemize}
\end{frame}


\begin{frame}[fragile]{Descriptive Statistics}
\textbf{Basic Statistics:}
\begin{itemize}
    \item Functions in R:
    \begin{verbatim}
min(dataName$ColumnName)
max(dataName$ColumnName)
mean(dataName$ColumnName)
summary(dataName$ColumnName)
    \end{verbatim}
\end{itemize}
\end{frame}
\begin{frame}[fragile]{Data Frames and Descriptive Statistics in R}
\textbf{Creating Data Frames:}
\begin{itemize}
    \item Use \texttt{data.frame()} to create a data frame.
    \begin{verbatim}
# Example:
newData <- data.frame(quakes[quakes[, 5] < 40, ])
    \end{verbatim}
    \item This example subsets rows from the \texttt{quakes} dataset where column 5 is less than 40.
\end{itemize}

\textbf{Quantile:}
\begin{itemize}
    \item Use \texttt{quantile(x, probs)} to calculate specific quantiles:
    \begin{verbatim}
quantile(vector, c(0.25, 0.5, 0.75))  # Q1, Median, Q3
    \end{verbatim}
\end{itemize}

\textbf{Five-Number Summary:}
\begin{itemize}
    \item Use \texttt{fivenum(data)} to compute:
        - Minimum, lower quartile, median, upper quartile, and maximum.
\end{itemize}
\end{frame}

\begin{frame}[fragile]{Sorting and Subsetting Data}
\textbf{Sorting Data:}
\begin{itemize}
    \item Use \texttt{sort()} to sort data:
    \begin{verbatim}
sortedData <- sort(variable, decreasing=TRUE)  # Descending order
    \end{verbatim}
\end{itemize}

\textbf{Subsetting Columns:}
\begin{itemize}
    \item Select specific columns using:
    \begin{verbatim}
quakes[, 2]  # Select column 2
    \end{verbatim}
\end{itemize}

\textbf{Subsetting Rows:}
\begin{itemize}
    \item Use logical conditions to filter rows:
    \begin{verbatim}
newVar <- oldData[oldData[, 5] < 40, ]
    \end{verbatim}
    \item This example subsets rows where column 5 is less than 40.
\end{itemize}
\end{frame}

\begin{frame}[fragile]{Data Visualization}
\textbf{Histograms:}
\begin{itemize}
    \item Create a histogram:
    \begin{verbatim}
hist(dataName$ColumnName, main="Title",
     xlab="X-axis Label", ylab="Y-axis Label")
    \end{verbatim}
\end{itemize}

\textbf{Boxplots:}
\begin{itemize}
    \item Create a boxplot:
    \begin{verbatim}
boxplot(dataName$ColumnName, main="Title",
        xlab="X-axis Label", ylab="Y-axis Label")
    \end{verbatim}
\end{itemize}
\end{frame}

\begin{frame}[fragile]{Multiple Plots and Axis Control}
\textbf{Multiple Plots:}
\begin{itemize}
    \item Use \texttt{par()} with \texttt{mfrow} for grid layout:
    \begin{verbatim}
par(mfrow=c(1, 2))
hist(dataName$ColumnName)
boxplot(dataName$ColumnName)
    \end{verbatim}
\end{itemize}

\textbf{Adjusting Axes:}
\begin{itemize}
    \item Use \texttt{xlim} and \texttt{ylim} to control axis scales:
    \begin{verbatim}
hist(dataName$ColumnName, ylim=c(minValue, maxValue))
    \end{verbatim}
\end{itemize}
\end{frame}

\begin{frame}{Lab Submissions}
\textbf{Requirements:}
\begin{itemize}
    \item Submit an R script (\texttt{.r}) with sufficient comments.
    \item Ensure the script runs without errors and produces the same output.
    \item Submit a report using the provided template on Canvas.
    \item Update your note sheet with summaries and new learnings.
\end{itemize}
\end{frame}

\end{document}
