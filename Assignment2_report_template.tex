\documentclass{article}
\usepackage{geometry}
\geometry{a4paper, margin=1in}
\usepackage{graphicx}
\usepackage{array}
\usepackage{booktabs}

\title{STAT 3611 Lab 2: Report Template}
\author{First and Last Name}
\date{}

\begin{document}

\maketitle

\section*{Trees Data Overview}
There are \_\_\_ rows and \_\_\_ columns in the Trees data set. The column headers are: 

\vspace{0.5cm}

Based on the documentation, the column originally named \_\_\_\_ has been renamed to \_\_\_\_\_.

\section*{Summarizing the Trees Data}
\textbf{Summary Statistics}

\begin{tabular}{@{}lcccc@{}}
\toprule
Statistic         & Column 1 & Column 2 & Column 3 \\ \midrule
Minimum Value     &          &          &          \\
First Quartile    &          &          &          \\
Median            &          &          &          \\
Third Quartile    &          &          &          \\
Maximum Value     &          &          &          \\
Mean              &          &          &          \\
Variance          &          &          &          \\ \bottomrule
\end{tabular}

\vspace{1cm}

\section*{Visual Summaries of the Trees Data}
*\textbf {For each of the plots, make sure that your plots have axis labels and a chart title.  Write a 1-2 sentence summary of your observations about the shape of the distribution for each column of data.}\\

\textbf{Boxplot and Histogram of Column 1:} \_\_\_\_

\vspace{0.5cm}
\textbf{Boxplot and Histogram of Column 2:} \_\_\_\_

\vspace{0.5cm}
\textbf{Boxplot and Histogram of Column 3:} \_\_\_\_

\section*{Data for Short Trees}
There are \_\_\_ trees with a height that is no greater than 65. 

\vspace{0.5cm}

\textbf{Table of Short Trees:}

\begin{tabular}{@{}llccc@{}}
\toprule
Tree ID & Girth & Height & Volume \\ \midrule
        &       &        &        \\
        &       &        &        \\ \bottomrule
\end{tabular}

\vspace{1cm}

\section*{Books Data Overview}
There are \_\_\_ rows and \_\_\_ columns in the Books data set. The column headers are: 

\vspace{0.5cm}

\section*{Number of Pages in the Books}
\textbf{Histogram:}

\begin{itemize}
    \item Create a histogram with a descriptive title to show the number of pages in the books.
\end{itemize}

There are \_\_\_ books with 0 pages.

\vspace{0.5cm}

\textbf{Outlier Threshold for Pages:}
\begin{tabular}{@{}lc@{}}
\toprule
Statistic           & Value \\ \midrule
First Quartile      &       \\
Third Quartile      &       \\
Interquartile Range &       \\
Upper Outlier Threshold & \\ \bottomrule
\end{tabular}

\vspace{0.5cm}

\textbf{Boxplot:}
\begin{itemize}
    \item Create a boxplot with a descriptive title showing the number of pages for books with at least one page. Adjust the axes to exclude outlier values.
\end{itemize}

\end{document}
