\documentclass{beamer}
\usepackage{graphicx}
\usepackage{amsmath}
\usepackage{xcolor}
\usetheme{Madrid}

\title[STAT 3611]{LAB 13: Goodness of Fit, Homogeneity, Independence}
\author{Kaosar}
\institute{Auburn University}
\date{Spring 2025}

\begin{document}

\begin{frame}
    \titlepage
\end{frame}

\begin{frame}[fragile]{Making \& Adding to Vectors Using \texttt{c()}}
    \begin{itemize}
        \item Make the original list, maybe with data in it:
        \begin{verbatim}
        friends <- c("Jason", "Sarah", "Jen")
        \end{verbatim}
        \item Add a new entry to the vector:
        \begin{verbatim}
        friends <- c(friends, "Kaley")
        \end{verbatim}
    \end{itemize}
\end{frame}

\begin{frame}[fragile]{Creating Category Columns}
    \begin{itemize}
    \item You have to have decided what the criteria are first
    \item Make a new column using the mutate function
    \item Embed the case\_when function to define the categories
        \item Use \texttt{mutate} and \texttt{case\_when}:
        \begin{verbatim}
        diamond2 <- diamond2 %>% mutate(
            PriceGroup = case_when(
                price <= 400 ~ "Low Price",
                price > 400 ~ "High Price"
            )
        )
        \end{verbatim}
    \end{itemize}
\end{frame}

\begin{frame}{Chi-Squared Goodness-of-Fit Tests}
    \begin{itemize}
        \item Tests if sample data fits a known distribution.
        \item Null Hypothesis: Data fits the distribution.
        \item Alternative Hypothesis: Data does not fit the distribution.
    \end{itemize}
\end{frame}
\begin{frame}[fragile]{GOF Test Example in R}
\begin{verbatim}
observed <- c(20, 15, 30, 35)
expected <- c(25, 25, 25, 25)
chisq.test(x = observed, p = expected/sum(expected))
\end{verbatim}
\end{frame}

\begin{frame}{Tests for Homogeneity}
    \begin{itemize}
        \item We want to know if multiple sets of data come from the same distribution.
        \item We do not have to know what that distribution is?
        \item Null Hypothesis: All groups have the same distribution.
        \item Alternative Hypothesis: At least one group differs.
    \end{itemize}
\end{frame}
\begin{frame}[fragile]{Homogeneity Test Example in R}
\begin{verbatim}
data <- matrix(c(30, 10, 20, 40), nrow = 2)
chisq.test(data)
\end{verbatim}
\end{frame}
\begin{frame}{Tests for Independence}
    \begin{itemize}
        \item Checks independence of two categorical variables.
        \item Null Hypothesis: Variables are independent.
        \item Independence:
        \[ \text{Pr}(A \cap B) = \text{Pr}(A) \times \text{Pr}(B) \]
    \end{itemize}
\end{frame}
\begin{frame}[fragile]{Independence Test Example in R}
\begin{verbatim}
data <- matrix(c(12, 5, 8, 15), nrow = 2)
chisq.test(data)
\end{verbatim}
\end{frame}
\begin{frame}{Homogeneity vs. Independence}
    \begin{itemize}
        \item “If you're thinking, "homogeneity and independence sound the same!", you're nearly right. The difference is a matter of design. 
        \item \textbf{Independence}: Data from one population, two variables observed per unit.
        \item \textbf{Homogeneity}: Data collected separately from multiple groups.
        \item The difference is subtle but crucial.
    \end{itemize}
\end{frame}

\begin{frame}[fragile]{The \texttt{chisq.test()} Function}
    \begin{itemize}
        \item Used for all three tests.
        \item The set of parameters for homogeneity and independence are the same.
        \item Check parameters carefully:
        \begin{verbatim}
        ?chisq.test
        \end{verbatim}
    \end{itemize}
\end{frame}

\begin{frame}[fragile]{Making Tables}
    \begin{itemize}
        \item Two options:
        \begin{verbatim}
        table()
        tabyl()
        \end{verbatim}
        \item Choose based on preference; neither is superior.
    \end{itemize}
\end{frame}
\begin{frame}[fragile]{Making Tables}
\begin{verbatim}
# Using table()
table(gender, response)

# Using tabyl() from janitor
library(janitor)
df %>% tabyl(gender, response)
\end{verbatim}
\end{frame}
\begin{frame}[fragile]{Anticipated Warning}
    \begin{verbatim}
    Warning message:
    Chi-squared approximation may be incorrect
    \end{verbatim}
    \begin{itemize}
        \item Common in small datasets; acceptable for this lab.
    \end{itemize}
\end{frame}

\begin{frame}[fragile]{Extracting Results from \texttt{chisq.test()}}
\begin{verbatim}
chisqResults <- chisq.test(myData)
pvalue <- chisqResults$p.value
statistic <- chisqResults$statistic
\end{verbatim}
\end{frame}


\begin{frame}{Lab Submissions}
    \begin{itemize}
        \item R script must be commented and executable.
        \item Report using Canvas template.
        \item Include all necessary descriptions and summaries.
        \item Submit updated note sheet.
    \end{itemize}
\end{frame}

\end{document}