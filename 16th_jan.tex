\documentclass{beamer}
\usepackage{graphicx}
\usepackage{amsmath}
\usetheme{Madrid}


\title{Week 1 (Lab 2): Introduction to R}
\author{Kaosar}
\institute{Auburn University}
\date{Spring 2025}

\begin{document}

\frame{\titlepage}

\begin{frame}{Overview of Lab 2}
\textbf{Topics Covered:}
\begin{itemize}
    \item Basic R Syntax
    \item Creating and Manipulating Vectors
    \item Sequences and Replication
    \item  Vectors and Logical Operators
    \item Handling Missing and Special Values
    \item Subsetting and Indexing
    \item Matrices and Data Frames
\end{itemize}
\end{frame}

\begin{frame}[fragile]{Basic R Syntax}
\textbf{Key Concepts:}
\begin{itemize}
    \item \texttt{>} Symbol is used to start a line in R.
    \item Use \texttt{<-} to assign values to variables (e.g., \texttt{x <- 5}).
    \item Arithmetic operations: \texttt{+, -, *, /, \textasciicircum}
    \item Up arrow recalls previous commands.
    \item If a vector is shorter than others in operations, it will recycle values.
\end{itemize}
\textbf{Example:}
\begin{verbatim}
x <- c(1.1, 9, 3.14)
\end{verbatim}
\end{frame}

\begin{frame}[fragile]{Creating Sequences}
\textbf{Using the \texttt{:} Operator:}
\begin{itemize}
    \item Creates sequences in increments of 1 (e.g., \texttt{1:10}).
\end{itemize}
\textbf{Using \texttt{seq()}:}
\begin{verbatim}
seq(0, 10, by=0.5)
seq(5, 10, length=30)
\end{verbatim}
\textbf{Replicating Values with \texttt{rep()}:}
\begin{verbatim}
rep(0, times=40)
rep(c(0, 1, 2), each=10)
\end{verbatim}
\end{frame}

\begin{frame}[fragile]{Vectors and Logical Operators}
\textbf{Vector Types:}
\begin{itemize}
    \item Atomic: numeric, logical, character, integer, complex.
    \item Lists: Can contain mixed data types.
\end{itemize}
\textbf{Logical Operators:}
\begin{itemize}
    \item \texttt{<, >, <=, >=, ==, !=}
    \item \texttt{\&} (AND), \texttt{|} (OR), \texttt{!} (NOT).
\end{itemize}
\textbf{Example:}
\begin{verbatim}
x <- c(10, 20, 30)
x[x > 15]  # Output: 20, 30
\end{verbatim}
\end{frame}


\begin{frame}[fragile]{Handling Missing and Special Values}
\textbf{NA and NAN:}
\begin{itemize}
    \item \texttt{NA}: Represents missing or unavailable values.
    \item \texttt{NAN}: Stands for "Not a Number" (e.g., division by zero).
\end{itemize}
\textbf{Example:}
\begin{verbatim}
x <- c(1, NA, 3)
is.na(x)  # Output: TRUE for missing values
\end{verbatim}
\end{frame}

\begin{frame}[fragile]{Subsetting and Indexing}
\textbf{Indexing Vectors:}
\begin{itemize}
    \item Use brackets \texttt{[]} to extract elements.
    \item Example: \texttt{x[1:10]} selects the first 10 elements.
\end{itemize}
\textbf{Naming Vectors:}
\begin{verbatim}
names(x) <- c("name1", "name2", "name3")
\end{verbatim}
\textbf{Example:}
\begin{verbatim}
x[c(2, 10)]   # Extracts the 2nd and 10th elements
x[-c(2, 10)]  # Excludes the 2nd and 10th elements
\end{verbatim}
\end{frame}




\begin{frame}[fragile]{Combining Characters}
\textbf{Using \texttt{paste()}:}
\begin{itemize}
    \item Joins characters together.
    \item \texttt{collapse=" "}: Specifies a space between elements.
    \item \texttt{sep()}: Specifies how to separate elements in the combined string.
\end{itemize}
\textbf{Examples:}
\begin{verbatim}
# Combine with collapse
paste(c("my", "name", "is"), collapse=" ")
# Output: "my name is"

# Combine with sep
paste("my", "name", "is", sep="-")
# Output: "my-name-is"
\end{verbatim}
\end{frame}

\begin{frame}[fragile]{Practical Examples}
\textbf{Task:}
\begin{itemize}
    \item Extract the 2nd and 10th elements from a vector.
    \item Subset a vector based on a condition.
\end{itemize}
\textbf{Solution:}
\begin{verbatim}
x <- c(2, 8, 3, 10)
x[c(2, 4)]  # Output: 8, 10
x[x > 5]   # Output: 8, 10
\end{verbatim}
\end{frame}

\begin{frame}[fragile]{Matrices and Data Frames}
\textbf{Key Functions:}
\begin{itemize}
    \item \texttt{matrix()} creates matrices.
    \item \texttt{cbind()} combines columns, \texttt{rbind()} combines rows.
    \item \texttt{dim()} sets dimensions of an object.
    \item \texttt{colnames()} assigns column names.
    \item \texttt{str()} summarizes data structure.
\end{itemize}

\end{frame}


\begin{frame}[fragile]{Creating Matrices: Key Functions}
\textbf{Key Functions:}
\begin{itemize}
    \item \texttt{matrix()} creates matrices.
    \item \texttt{dim()} sets or retrieves dimensions of a matrix.
\end{itemize}
\end{frame}

\begin{frame}[fragile]{Creating Matrices: Example}
\textbf{Example:}
\begin{verbatim}
# Create a matrix with 2 rows and 3 columns
m <- matrix(1:6, nrow=2, ncol=3)

# Add column names
dimnames(m) <- list(NULL, c("A", "B", "C"))
print(m)
\end{verbatim}
\textbf{Output:}
\begin{verbatim}
  A B C
[1,] 1 3 5
[2,] 2 4 6
\end{verbatim}
\end{frame}

\begin{frame}[fragile]{Retrieving Dimensions of a Matrix}
\textbf{Example:}
\begin{verbatim}
# Retrieve dimensions of the matrix
dim(m)  # Output: [1] 2 3
\end{verbatim}
\textbf{Output:}
\begin{verbatim}
[1] 2 3
\end{verbatim}
\end{frame}

\begin{frame}[fragile]{Combining Rows and Columns: Key Functions}
\textbf{Key Functions:}
\begin{itemize}
    \item \texttt{cbind()} combines columns to create or expand matrices.
    \item \texttt{rbind()} combines rows to create or expand matrices.
\end{itemize}
\end{frame}

\begin{frame}[fragile]{Combining Columns: Example}
\textbf{Example:}
\begin{verbatim}
# Combine vectors as columns
v1 <- c(1, 2, 3)
v2 <- c(4, 5, 6)
cbind_matrix <- cbind(v1, v2)
print(cbind_matrix)
\end{verbatim}
\textbf{Output:}
\begin{verbatim}
     v1 v2
[1,]  1  4
[2,]  2  5
[3,]  3  6
\end{verbatim}
\end{frame}

\begin{frame}[fragile]{Combining Rows: Example}
\textbf{Example:}
\begin{verbatim}
# Combine vectors as rows
rbind_matrix <- rbind(v1, v2)
print(rbind_matrix)
\end{verbatim}
\textbf{Output:}
\begin{verbatim}
     [,1] [,2] [,3]
[1,]    1    2    3
[2,]    4    5    6
\end{verbatim}
\end{frame}

\begin{frame}[fragile]{Data Frames: Key Functions}
\textbf{Key Functions:}
\begin{itemize}
    \item \texttt{data.frame()} creates data frames.
    \item \texttt{str()} summarizes data structure.
    \item \texttt{dim()} retrieves dimensions.
    \item \texttt{colnames()} assigns column names.
\end{itemize}
\end{frame}

\begin{frame}[fragile]{Creating and Summarizing Data Frames}
\textbf{Example:}
\begin{verbatim}
# Create a data frame
df <- data.frame(Name=c("Alice", "Bob"), Age=c(25, 30))

# Check structure
str(df)
\end{verbatim}
\textbf{Output:}
\begin{verbatim}
'data.frame': 2 obs. of  2 variables:
 $ Name: chr  "Alice" "Bob"
 $ Age : num  25 30
\end{verbatim}
\end{frame}

\begin{frame}[fragile]{Retrieving Dimensions of a Data Frame}
\textbf{Example:}
\begin{verbatim}
# Retrieve dimensions of the data frame
dim(df)  # Output: [1] 2 2
\end{verbatim}
\textbf{Output:}
\begin{verbatim}
[1] 2 2
\end{verbatim}
\end{frame}




\end{document}
