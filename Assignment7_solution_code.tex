\documentclass{article}

\usepackage{listings}
\usepackage{xcolor}

\lstset{
  language=R,
  basicstyle=\ttfamily\small,
  keywordstyle=\color{blue},
  commentstyle=\color{green!60!black},
  stringstyle=\color{red},
  frame=single,
  breaklines=true,
  numbers=left,
  numberstyle=\tiny,
  captionpos=b
}

\begin{document}

\subsection*{STAT 3611 Lab 7 Solution}

\subsection*{3.1 Initial Data Overview}
\begin{lstlisting}[language=R]
data("faithful")
colnames(faithful)
nrow(faithful)
\end{lstlisting}
Answer:
- Columns: \texttt{"eruptions"}, \texttt{"waiting"}
- Number of rows: \texttt{272}

\subsection*{3.2 Summary Stats for the Full Dataset}
\begin{lstlisting}[language=R]
eruptions <- faithful$eruptions
waiting <- faithful$waiting

mean(eruptions)
var(eruptions) * (length(eruptions) - 1) / length(eruptions)
sqrt(var(eruptions) * (length(eruptions) - 1) / length(eruptions))
mean(eruptions >= 4)
\end{lstlisting}
Answer:
- Computed mean, variance, standard deviation, and proportion of eruptions $\geq 4$.

\subsection*{3.3 Sampling}
\begin{lstlisting}[language=R]
set.seed(123)
samples <- replicate(100, sample(faithful$eruptions, 10, replace = TRUE))
samples_df <- data.frame(samples)
\end{lstlisting}
Answer:
- Created a data frame containing 100 samples of size 10 from eruption durations.

\subsection*{3.4 Analyze the Samples}
\begin{lstlisting}[language=R]
sample_means <- colMeans(samples_df)
sample_vars <- apply(samples_df, 2, var)
sample_sds <- sqrt(sample_vars)
p_hat <- apply(samples_df, 2, function(x) mean(x >= 4))

mean(sample_means)
mean(sample_vars)
mean(sample_sds)
mean(p_hat)

bias_means <- mean(faithful$eruptions) - mean(sample_means)
bias_vars <- var(faithful$eruptions) - mean(sample_vars)
bias_sds <- sd(faithful$eruptions) - mean(sample_sds)
bias_p_hat <- mean(faithful$eruptions >= 4) - mean(p_hat)
\end{lstlisting}
Answer:
- Calculated sample means, variances, standard deviations, and proportion estimates (p-hat) for each sample.
- Computed average values and biases (difference between true population parameters and their estimates) for sample means, variances, standard deviations, and proportions.

\subsection*{3.5 Visual Summaries}
\begin{lstlisting}[language=R]
boxplot(sample_vars, main="Boxplot of Sample Variances", ylab="Variance")
boxplot(p_hat, main="Boxplot of Proportion Estimates (P-hat)", ylab="P-hat")
\end{lstlisting}
Answer:
- Created boxplots for sample variances and proportion estimates.

\subsection*{3.6 Distribution Quantiles}
\begin{lstlisting}[language=R]
qchisq(0.975, df=9)
qnorm(0.95)
\end{lstlisting}
Answer:
- Calculated chi-squared and z quantiles for confidence intervals.

\subsection*{3.7 Confidence Intervals}
\begin{lstlisting}
chi_lower <- (9 * sample_vars) / qchisq(0.975, df=9)
chi_upper <- (9 * sample_vars) / qchisq(0.025, df=9)
contains_sd <- (sqrt(chi_lower) <= sd(eruptions)) & (sqrt(chi_upper) >= sd(eruptions))
sum(contains_sd)

z_value <- qnorm(0.95)
p_hat_upper <- p_hat + z_value * sqrt(p_hat * (1 - p_hat) / 10)
contains_prop <- mean(eruptions >= 4) <= p_hat_upper
sum(contains_prop)
\end{lstlisting}
Answer:
- Calculated 95 confidence intervals for standard deviation and proportion parameters. Counted intervals containing the true standard deviation and proportion.

\subsection*{3.8 Data Frame Modification}
\begin{lstlisting}
samples_df <- rbind(samples_df,
                    sample_sds,
                    sqrt(chi_lower),
                    sqrt(chi_upper),
                    contains_sd,
                    p_hat,
                    p_hat_upper,
                    contains_prop)

row.names(samples_df)[11:17] <- c("Sample_SD", "SD_CI_Lower", "SD_CI_Upper", "Contains_SD", "P-hat", "P-hat_CI_Upper", "Contains_Prop")
\end{lstlisting}
Answer:
- Appended the confidence interval calculations, sample statistics, and proportion parameters into the data frame and renamed rows with descriptive labels.

\subsection*{3.9 Write Data Frame to CSV}
\begin{lstlisting}[language=R]
write.csv(samples_df, "Lab7_LastNameFirstInitial.csv", row.names=TRUE)
\end{lstlisting}
Answer:
- Exported data frame to CSV.

\end{document}
