\documentclass{article}
\usepackage{graphicx}
\usepackage{booktabs}
\usepackage{hyperref}
\usepackage{float}

\title{STAT 3611 - Lab Test 1 Report}
\author{Your Name}
\date{\today}

\begin{document}

\maketitle

\section{Dataset 1: ToothGrowth}

\subsection{Q1: Initial Data Overview}
\begin{itemize}
    \item The dataset \texttt{ToothGrowth} is loaded in R.
    \item The column headers are: \---, \---, and \---.
    \item The dataset contains \texttt{XX} rows.
    \item There are \texttt{XX} different supplements in this dataset.
\end{itemize}

\subsection{Q2: Summary Statistics}
For each supplement (VC and OJ), the following summary statistics are computed:

\begin{table}[H]
    \centering
    \begin{tabular}{lcc}
        \toprule
        Statistic & Supplement VC & Supplement OJ \\
        \midrule
        Minimum & XX & XX \\
        First Quartile (Q1) & XX & XX \\
        Median & XX & XX \\
        Mean & XX & XX \\
        Third Quartile (Q3) & XX & XX \\
        Maximum & XX & XX \\
        Interquartile Range (IQR) & XX & XX \\
        Variance & XX & XX \\
        \bottomrule
    \end{tabular}
    \caption{Summary statistics for tooth length by supplement type.}
    \label{tab:summary_toothgrowth}
\end{table}

\subsection{Q3: Data Visualization}

\begin{figure}[H]
    \centering
    \includegraphics[width=0.8\textwidth]{boxplots_toothgrowth.png} % Update with correct filename
    \caption{Boxplots of Tooth Length for Supplements VC and OJ}
    \label{fig:boxplots_toothgrowth}
\end{figure}

\begin{figure}[H]
    \centering
    \includegraphics[width=0.8\textwidth]{histograms_toothgrowth.png} % Update with correct filename
    \caption{Histograms of Tooth Length for Supplements VC and OJ}
    \label{fig:histograms_toothgrowth}
\end{figure}

\textbf{Explanation:} Based on the boxplots and histograms, the differences between the tooth length obtained by supplement VC and OJ can be summarized as follows: \textit{(Provide a short interpretation here based on your analysis.)}

\section{Dataset 2: Palmer Penguins}

\subsection{Q4: Initial Data Overview}
\begin{itemize}
    \item The dataset \texttt{penguins} is loaded in R.
    \item The column headers are: \---, \---, \---.
    \item The dataset contains \texttt{XX} rows and \texttt{XX} columns.
    \item The unique species in the dataset are: \texttt{XX, XX, XX}.
    \item After sorting alphabetically, the first two species are \texttt{XX, XX} and the last two species are \texttt{XX, XX}.
\end{itemize}

\subsection{Q5: Body Mass Analysis}

\begin{itemize}
    \item A new dataframe was created with penguins having body mass \texttt{$\geq$ 4000 g}.
    \item The dataset was cleaned to remove missing values.
    \item The total number of penguins with body mass \texttt{$\geq$ 4000 g} and no missing values is \texttt{XX}.
\end{itemize}

\begin{figure}[H]
    \centering
    \includegraphics[width=0.8\textwidth]{histogram_penguins_mass.png} % Update with correct filename
    \caption{Histogram of Body Mass for Penguins with at least 4000g}
    \label{fig:histogram_penguins_mass}
\end{figure}

\begin{figure}[H]
    \centering
    \includegraphics[width=0.8\textwidth]{qqplot_penguins_mass.png} % Update with correct filename
    \caption{QQ Plot of Body Mass for Penguins with at least 4000g}
    \label{fig:qqplot_penguins_mass}
\end{figure}

\textbf{Explanation:} The histogram and QQ plot indicate that the distribution of body mass is \textit{(Provide a short interpretation here based on your analysis.)}

\section{Conclusion}
In this lab test, we analyzed the \texttt{ToothGrowth} and \texttt{penguins} datasets using descriptive statistics and visualizations in R. Our analysis showed \textit{(Summarize key findings here.)}

\end{document}
