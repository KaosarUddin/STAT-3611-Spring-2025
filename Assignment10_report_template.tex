\documentclass{article}
\usepackage{graphicx}
\usepackage{booktabs}
\usepackage{hyperref}
\usepackage{float}

\title{STAT 3611 - Lab 10 Report}
\author{Your First and Last Name}
\date{Submission Date}

\begin{document}

\maketitle

\section{Introduction}
This report presents the analysis conducted in Lab 10, which includes exploratory data analysis, summary statistics, t-tests, and hypothesis testing for proportions using the \texttt{birthwt} and \texttt{precip} datasets in R.

\section{Initial Data Overview: birthwt Dataset}
\begin{itemize}
    \item The dataset \texttt{birthwt} was loaded using the MASS package.
    \item Column headers: \textit{[list them here]}.
    \item Total number of rows: \_\_
    \item A new column for birthweight in pounds was added using the conversion 454 grams = 1 pound.
    \item Two new data frames were created:
    \begin{itemize}
        \item One for babies born to mothers who smoked.
        \item One for babies born to mothers who did not smoke.
    \end{itemize}
    \item For each group, the following values were computed (in both grams and pounds):
    \begin{itemize}
        \item Sample size
        \item Sample mean
        \item Sample variance
        \item Skewness (computed using the \texttt{e1071} package)
    \end{itemize}
    \item Based on the summary statistics, \textit{we expect/do not expect} the mean birthweight to be the same. (Brief explanation)
\end{itemize}

\section{Two-Sample t-Test (Unknown Variance, \(\alpha = 0.05\))}
\subsection{QQ Plots}
QQ plots for birthweight (grams) are shown below:

\begin{figure}[H]
    \centering
    \includegraphics[width=0.9\textwidth]{qqplots_birthwt.png} % Update with your actual image file
    \caption{QQ plots of birthweights: all babies, smokers, and non-smokers.}
    \label{fig:qqplots}
\end{figure}

\subsection{Two-sided t-test}
\begin{itemize}
    \item Null Hypothesis: 
    \item Alternative Hypothesis: 
    \item Result of \texttt{t.test()}: \textit{[insert output summary]}.
    \item Conclusion: Reject/Fail to Reject Null Hypothesis. \textit{(Brief explanation)}
\end{itemize}

\subsection{One-sided t-test}
\begin{itemize}
    \item Null Hypothesis: 
    \item Alternative Hypothesis: 
    \item Result of \texttt{t.test()}: \textit{[insert output summary]}.
    \item Conclusion: Reject/Fail to Reject Null Hypothesis. \textit{(Brief explanation)}
\end{itemize}

\section{Hypothesis Test for the Proportion Parameter: precip Dataset}
\begin{itemize}
    \item The \texttt{precip} dataset was loaded.
    \item A sample of size 25 was taken using \texttt{sample()}.
    \item Proportion of cities with rainfall \(\geq 20\) inches was tested:
    \begin{itemize}
        \item Null Hypothesis: 
        \item Alternative Hypothesis: 
        \item Significance level: 
        \item Method used: \texttt{prop.test()} or \texttt{binom.test()}
        \item Conclusion: \textit{Reject/Fail to Reject Null Hypothesis}
    \end{itemize}
    \item One-sided test for whether \(p > 0.65\) or \(p < 0.65\):
    \begin{itemize}
        \item Null Hypothesis: 
        \item Alternative Hypothesis: 
        \item Conclusion: \textit{Reject/Fail to Reject Null Hypothesis}
    \end{itemize}
    \item True proportion of cities in the full dataset with rainfall over 20 inches: \_\_
    \item Based on this true proportion, the test result is reasonable/unexpected. \textit{(Explanation)}
\end{itemize}

\section{Conclusion}
Summarize the major findings from Lab 10. Discuss the difference in birthweight distributions between smokers and non-smokers, the conclusions drawn from proportion tests, and any challenges or insights.

\section{References}
List any references used, such as R documentation, package help files, or course materials.

\end{document}
