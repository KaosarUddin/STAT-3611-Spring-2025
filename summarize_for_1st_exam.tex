\documentclass{article}
\usepackage{graphicx}
\usepackage{amsmath}
\usepackage{hyperref}
\usepackage{listings}
\usepackage{xcolor}

\lstset{
  language=R,
  basicstyle=\ttfamily\footnotesize,
  keywordstyle=\color{blue},
  commentstyle=\color{green},
  stringstyle=\color{red},
  numbers=left,
  numberstyle=\tiny,
  stepnumber=1,
  breaklines=true,
  showstringspaces=false
}

\begin{document}

\title{Review for Test 1}
\author{Kaosar, Auburn University}
\date{Spring 2025}
\maketitle


This document summarizes key concepts from the STAT 3611 lab sessions ( week 1 to week 4 and assignments 2-4), covering fundamental R programming techniques, data manipulation, visualization, and statistical analysis.

\section{Week 1: Introduction to R}
\begin{itemize}
    \item Basic R syntax: Using \texttt{<-} for assignments, arithmetic operations, and command recall.
    \item Creating and manipulating vectors: \texttt{c()}, \texttt{seq()}, \texttt{rep()}.
    \item Logical operators: \texttt{<, >, ==, !=, \&, |, !}.
    \item Handling missing values: \texttt{NA}, \texttt{NaN}, and functions like \texttt{is.na()}.
    \item Subsetting vectors: Using brackets \texttt{[]} and negative indices for exclusion.
    \item Matrices and data frames: Functions \texttt{matrix()}, \texttt{data.frame()}, \texttt{colnames()}.
\end{itemize}



\section{Week 2: Data Description in R}
\begin{itemize}
    \item One-based indexing in R.
    \item Accessing help: \texttt{help()}, \texttt{?}.
    \item Exploring datasets: \texttt{dim()}, \texttt{nrow()}, \texttt{ncol()}, \texttt{head()}.
    \item Subsetting data: \texttt{subset()}, filtering with logical conditions.
    \item Descriptive statistics: \texttt{min()}, \texttt{max()}, \texttt{mean()}, \texttt{summary()}, \texttt{quantile()}.
    \item Data visualization: Histograms and boxplots using \texttt{hist()} and \texttt{boxplot()}.
\end{itemize}



\section{Week 3: Working with CSV Files and Functions}
\begin{itemize}
    \item Reading CSV files: \texttt{read.csv()}.
    \item Creating user-defined functions.
    \item Subsetting and sorting data: \texttt{sort()}, \texttt{unique()}.
    \item Order statistics: \texttt{median()}, \texttt{quantile()}.
\end{itemize}



\section{Week 4: Advanced Data Handling and Visualization}
\begin{itemize}
    \item Using \texttt{unique()} to remove duplicates.
    \item Checking for membership: \texttt{\%in\%} operator.
    \item Installing and loading packages: \texttt{install.packages()}, \texttt{library()}.
    \item Regular expressions with \texttt{str\_detect()} from \texttt{stringr} package.
    \item Correlation: \texttt{cor()} function.
    \item QQ plots: \texttt{qqnorm()}, \texttt{qqline()}.
\end{itemize}


\section{Assignment Highlights}
\subsection{Assignment 2: Trees and Books Dataset}
\begin{itemize}
    \item Renaming columns in data frames.
    \item Subsetting data based on conditions.
    \item Computing summary statistics and creating visualizations.
\end{itemize}

\subsection{Assignment 3: Quakes Dataset}
\begin{itemize}
    \item Summary statistics: Mean, variance, standard deviation, quartiles.
    \item Data visualization: Histograms and boxplots.
    \item Subsetting based on station count.
    \item Sorting and order statistics.
\end{itemize}



\subsection{Assignment 4: Books Dataset}
\begin{itemize}
    \item Using \texttt{str\_detect()} to filter text data.
    \item Finding percentiles and visualizing data distributions.
    \item Computing correlation between variables.
    \item Creating QQ plots to assess normality.
\end{itemize}



\end{document}
