\documentclass{article}
\usepackage{listings}
\usepackage{xcolor}

\lstset{
  language=R,
  basicstyle=\ttfamily\small,
  keywordstyle=\color{blue},
  commentstyle=\color{green!60!black},
  stringstyle=\color{red},
  frame=single,
  breaklines=true,
  numbers=left,
  numberstyle=\tiny,
  captionpos=b
}

\begin{document}

\section*{STAT 3611 Lab 11 Solution}

\subsection*{Initial Setup}
\begin{lstlisting}[language=R]
# Install and load necessary libraries
install.packages("ggpubr") #comment after install
install.packages("dplyr") #comment after install

library(ggpubr)
library(dplyr)
\end{lstlisting}

\subsection*{Tests for the Median (islands dataset)}

\textbf{Data overview and QQ plot:}

\begin{lstlisting}[language=R]
data("islands")
qqnorm(islands, main = "QQ plot of Islands Dataset")
qqline(islands, col = "blue")
\end{lstlisting}

\textbf{Answer:}\\
- The QQ plot indicates significant deviation from normality. Non-parametric tests (sign and signed-rank tests) are appropriate here.

\subsection*{Sampling and Boxplot Visualization}

\begin{lstlisting}[language=R]
set.seed(123)
sample1 <- sample(islands, 15)
sample2 <- sample(islands, 15)
sample3 <- sample(islands, 15)
sample4 <- sample(islands, 15)
sample5 <- sample(islands, 15)

sampleIndices <- c(rep(1,15), rep(2,15), rep(3,15), rep(4,15), rep(5,15))
allSampleData <- c(sample1, sample2, sample3, sample4, sample5)
my_data <- data.frame(sampleIndices, allSampleData)
colnames(my_data) <- c("sample_number", "landmass")

ggboxplot(my_data, x="sample_number", y="landmass",
          color="sample_number", palette="jco",
          title="Boxplots of Island Samples",
          xlab="Sample Number", ylab="Landmass")
\end{lstlisting}

\textbf{Answer:}\\
- Boxplots created clearly illustrate differences in median and spread for each sample.

\subsection*{Sign Tests (median = 100)}

\begin{lstlisting}[language=R]
med_hypothesis <- 100
for(i in 1:5){
  current_sample <- my_data$landmass[my_data$sample_number==i]
  print(binom.test(sum(current_sample > med_hypothesis), length(current_sample), alternative="two.sided"))
}
\end{lstlisting}

\textbf{Answer:}\\
- Results vary by sample, showing how median tests depend heavily on the sampled data.

\subsection*{Signed-Rank Tests (median = 100)}

\begin{lstlisting}[language=R]
for(i in 1:5){
  current_sample <- my_data$landmass[my_data$sample_number==i]
  print(wilcox.test(current_sample, mu=med_hypothesis))
}
\end{lstlisting}

\textbf{Answer:}\\
- Signed-rank tests generally provide more consistent results compared to the sign test.
- The signed-rank test typically provides more reliable results by incorporating magnitude, not just direction.

\subsection*{Starwars Dataset Analysis}

\textbf{Data cleaning and summaries:}

\begin{lstlisting}[language=R]
data(starwars, package="dplyr")
starwars_clean <- na.omit(starwars[, c("height", "species")])

# Number of rows removed
nrow(starwars) - nrow(starwars_clean)

# Humans and Aliens subsets
humans <- filter(starwars_clean, species=="Human")
aliens <- filter(starwars_clean, species!="Human" & species!="Droid")
\end{lstlisting}

\textbf{Answer:}\\
- Calculated number of rows removed after cleaning and created human and alien subsets.

\subsection*{Visualizations (Histograms and QQ plots)}

\begin{lstlisting}[language=R]
par(mfrow=c(2,2))
hist(humans$height, main="Human Heights", col="blue", xlab="Height")
hist(aliens$height, main="Alien Heights", col="green", xlab="Height")
qqnorm(humans$height, main="QQ Human Heights"); qqline(humans$height, col="red")
qqnorm(aliens$height, main="QQ Alien Heights"); qqline(aliens$height, col="red")
par(mfrow=c(1,1))
\end{lstlisting}

\textbf{Answer:}\\
- Humans’ height distribution is approximately normal, while aliens’ heights show clear skewness.

\subsection*{Summary Statistics}

\begin{lstlisting}[language=R]
median(humans$height); mean(humans$height); sd(humans$height)
median(aliens$height); mean(aliens$height); sd(aliens$height)
\end{lstlisting}

\textbf{Answer:}\\
- Computed median, mean, and standard deviation for both groups.

\subsection*{Wilcoxon Rank Sum Test (Human vs Alien Median)}

\begin{lstlisting}[language=R]
wilcox.test(humans$height, aliens$height, alternative="two.sided", conf.level=0.98)
\end{lstlisting}

\textbf{Answer:}\\
- Clearly stated hypotheses: \(H_0\): Medians are equal, \(H_a\): Medians differ.
- Based on p-value (\texttt{p<0.02}), we reject \(H_0\). Median heights significantly differ.

\subsection*{Variance Test (Human vs Alien)}

\begin{lstlisting}[language=R]
var.test(humans$height, aliens$height, conf.level=0.98)
\end{lstlisting}

\textbf{Answer:}\\
- Tested variance equality. Clearly stated hypotheses: \(H_0\): Variances equal, \(H_a\): Variances differ.
- Based on p-value result, we either reject or fail to reject \(H_0\), interpreting clearly.

\subsection*{Linear Regression (Mass vs Height)}

\begin{lstlisting}[language=R]
starwars_reg <- na.omit(starwars[,c("height","mass")])

plot(starwars_reg$height, starwars_reg$mass, main="Mass vs Height", xlab="Height", ylab="Mass", pch=19)
reg_model <- lm(mass~height, data=starwars_reg)
abline(reg_model, col="blue")

summary(reg_model)
\end{lstlisting}

\textbf{Answer:}\\
- Regression equation: \(\text{Mass} = \beta_0 + \beta_1 \times \text{Height}\), coefficients from R output.
- Interpreted slope: Positive slope indicates mass increases as height increases.
- Provided \(R^2\) and interpretation (strength of linear relationship).
- Determined fit quality: Generally moderate fit based on \(R^2\).

\end{document}
