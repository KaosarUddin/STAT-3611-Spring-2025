\documentclass[12pt]{article}
\usepackage{amsmath}
\usepackage{listings}
\usepackage{xcolor}

% Define the listing settings
\lstset{
    language=R,
    basicstyle=\ttfamily\small,
    keywordstyle=\color{blue}\bfseries,   % Keywords in blue
    commentstyle=\color{green!50!black},  % Comments in green
    stringstyle=\color{orange},           % Strings in orange
    numbers=left,                          % Enable line numbers outside
    numberstyle=\tiny\color{gray},        % Line numbers in gray and tiny
    stepnumber=1,                          % Number every line
    numbersep=10pt,                        % Space between numbers and code
    xleftmargin=20pt,                      % Ensure space for line numbers
    frame=single,                          % Box around code
    showstringspaces=false,
    breaklines=true,
    postbreak=\mbox{\textcolor{red}{$\hookrightarrow$}\space}
}

\begin{document}

\title{STAT 3611 Lab Test 1 Solution}
\author{Auburn University}
\date{Spring 2025}
\maketitle

\section{Dataset 1: ToothGrowth}

\subsection{Problem 1: Initial Data Overview}
\begin{lstlisting}
data("ToothGrowth") # Load dataset
colnames(ToothGrowth) # Get column names
nrow(ToothGrowth) # Count rows
unique(ToothGrowth$supp) # Get unique supplement types
length(unique(ToothGrowth$supp)) # Count unique supplement types
\end{lstlisting}
Answer: Load the dataset, extract column headers, count rows, and find unique supplement types.

\subsection{Problem 2: Summary Statistics}
\begin{lstlisting}
VC_data <- subset(ToothGrowth, supp == "VC") # Subset VC data
OJ_data <- subset(ToothGrowth, supp == "OJ") # Subset OJ data
summary(VC_data$len) # Compute summary statistics for VC
summary(OJ_data$len) # Compute summary statistics for OJ
var(VC_data$len) # Compute variance for VC
var(OJ_data$len) # Compute variance for OJ
IQR(VC_data$len) # Compute IQR for VC
IQR(OJ_data$len) # Compute IQR for OJ
\end{lstlisting}
Answer: Compute summary statistics including variance and IQR.

\subsection{Problem 3: Visualization}
\begin{lstlisting}
par(mfrow = c(1, 2))
boxplot(VC_data$len, 
    main="Boxplot of Tooth Length (VC)",
    ylab="Tooth Length",
    col="blue")

boxplot(OJ_data$len, 
    main="Boxplot of Tooth Length (OJ)",
    ylab="Tooth Length",
    col="red")

par(mfrow = c(2, 1))
hist(VC_data$len, 
    main="Histogram of Tooth Length (VC)",
    xlab="Tooth Length",
    col="blue")

hist(OJ_data$len, 
    main="Histogram of Tooth Length (OJ)",
    xlab="Tooth Length",
    col="red")
\end{lstlisting}
Answer: Create boxplots and histograms for both supplement types.

\section{Dataset 2: Palmer Penguins}

\subsection{Problem 4: Initial Data Overview}
\begin{lstlisting}
install.packages("palmerpenguins") # Install package
library(palmerpenguins) # Load library
data("penguins") # Load dataset
colnames(penguins) # Get column names
dim(penguins) # Get dimensions (rows and columns)
unique(penguins$species) # Get unique species
sorted_species <- sort(unique(penguins$species)) # Sort species alphabetically
sorted_species # Display sorted species
sorted_species[c(1, 2, length(sorted_species)-1, length(sorted_species))] # First and last 2 species
\end{lstlisting}
Answer: Load dataset, extract column headers, and analyze species names.

\subsection{Problem 5: Body Mass Analysis}
\begin{lstlisting}
penguins_filtered <- subset(penguins, body_mass_g >= 4000) # Subset data
penguins_filtered <- na.omit(penguins_filtered) # Remove NA values
nrow(penguins_filtered) # Count filtered rows

par(mfrow = c(1, 1)) # Set plotting window
hist(penguins_filtered$body_mass_g, 
    main="Histogram of Body Mass (≥4000g)",
    xlab="Body Mass (g)",
    col="blue")

qqnorm(penguins_filtered$body_mass_g, 
    main="QQ Plot of Body Mass (≥4000g)")
qqline(penguins_filtered$body_mass_g, col="red")
\end{lstlisting}
Answer: Filter dataset, clean NA values, generate histogram and QQ plot.

\end{document}
