\documentclass{article}
\usepackage{listings}
\usepackage{xcolor}

\lstset{
  language=R,
  basicstyle=\ttfamily\small,
  keywordstyle=\color{blue},
  commentstyle=\color{green!60!black},
  stringstyle=\color{red},
  frame=single,
  breaklines=true,
  numbers=left,
  numberstyle=\tiny,
  captionpos=b
}

\begin{document}

\section*{STAT 3611 Lab 9 Solution Code}

\subsection*{1. Initial Data Overview}
\begin{lstlisting}
data("Nile")
length(Nile)

data("AirPassengers")
length(AirPassengers)
\end{lstlisting}
Answer:
- Nile dataset: \texttt{100} years
- AirPassengers dataset: \texttt{144} data points

\subsection*{2. Summary Statistics for the Nile Dataset}
\begin{lstlisting}
mean(Nile)
var(Nile)
qqnorm(Nile)
qqline(Nile, col="red")
\end{lstlisting}
Answer:
- Calculated mean and variance.
- QQ-plot indicates approximate normality as points closely follow the reference line.

\subsection*{3. 2-Sided Test Known Variance (without t.test())}
\begin{lstlisting}
# Hypotheses:
# H0: mu = 920
# Ha: mu != 920

alpha <- 0.05
z_value <- qnorm(1 - alpha / 2)
sigma <- sqrt(28350)
n <- length(Nile)

upper_rejection <- 920 + z_value * (sigma / sqrt(n))
lower_rejection <- 920 - z_value * (sigma / sqrt(n))

z_stat <- (mean(Nile) - 920) / (sigma / sqrt(n))
z_stat

# Decision:
(abs(z_stat) > z_value)

# Type II error calculation if true mean is 917:
pnorm(upper_rejection, mean=917, sd=sigma/sqrt(n)) - 
  pnorm(lower_rejection, mean=917, sd=sigma/sqrt(n))
\end{lstlisting}
Answer:
- Null hypothesis \texttt{rejected} due to z-statistic exceeding the critical value.
- Type II error computed for true mean \texttt{917}.

\subsection*{4. 2-Sided Test Known Variance (using t.test())}
\begin{lstlisting}
t.test(Nile, mu=920, alternative="two.sided", conf.level=0.95)
\end{lstlisting}
Answer:
- Null hypothesis \texttt{rejected}; result consistent with previous test.

\subsection*{5. 1-Sided Test Unknown Variance}
\begin{lstlisting}
# H0: mu = 280
# Ha: mu > 280

# Extract data for 1955:
start_1955 <- (1955 - 1949) * 12 + 1
end_1955 <- start_1955 + 11

data_1955 <- AirPassengers[start_1955:end_1955]
mean_1955 <- mean(data_1955)
sd_1955 <- sd(data_1955)
n_1955 <- length(data_1955)

# t-value calculation
t_value <- qt(0.95, df=n_1955-1)

# Perform test using t.test()
test_result <- t.test(data_1955, mu=280, alternative="greater")
test_result
\end{lstlisting}
Answer:
- Computed sample mean and standard deviation for 1955.
- p-value obtained from the t-test indicates to \texttt{reject} the null hypothesis at $\alpha = 0.05$.

\end{document}
