
\documentclass{article}
\usepackage{geometry}
\geometry{a4paper, margin=1in}
\usepackage{hyperref}
\usepackage{fancyhdr}
\pagestyle{fancy}
\usepackage{enumitem}
\usepackage{graphicx}
\fancyhead[L]{STAT 3611 Assignment (Lab 4)}
\fancyhead[R]{Feb 17 at 11:59pm}
\setlist[itemize]{topsep=2pt, itemsep=1pt, left=10pt}
\setlist[enumerate]{topsep=2pt, itemsep=1pt, left=10pt}


\begin{document}

\section*{Topic Overview}
In this lab you will practice the following skills in R:
\begin{itemize}
    \item Reading in data from a CSV file
    \item Using R libraries
    \item Isolating a subset of a data frame
    \item Computing summary statistics for a column of data
    \item Finding percentiles of a data set
    \item Creating histograms
    \item Creating boxplots
    \item Creating QQ plots
    \item \textbf{Optional:} Writing loops and functions in R
\end{itemize}

\section*{Submission Instructions}
For this lab, you will submit 3 files:
\begin{itemize}
    \item Your code file must have a \texttt{.r} extension and be named \texttt{Lab4\_LastNameFirstName.r}.
    \item Your lab report must have a \texttt{.pdf} extension and be named \texttt{Lab4\_LastNameFirstName.pdf}.
    \item Your updated reference note sheet must have a \texttt{.pdf} extension and be named \texttt{Lab4\_Notes\_LastNameFirstName.pdf}.
\end{itemize}
The reference sheet should contain all of the information from the previous labs and any additional notes about new functions/code blocks/tips you have acquired in this lab.

\section*{Lab Instructions}
This lab will use the data in the file \texttt{books.csv} that accompanies the lab instructions.

This lab will use the function \texttt{str\_detect()}, so you must tell R that you want to use the \texttt{stringr} library by including the code:
\begin{verbatim}
library(stringr)
\end{verbatim}
at the top of your script.

\textbf{You do not need to write your own function to find the mode,} you may use the one provided in the R file that accompanies this lab. To do so, copy the function code into your lab solution file, and it must be run before you call the function in your work.

\section*{1) Install and Load the stringr Package}
\subsection*{a. Install the stringr Package}
\textbf{Using the GUI interface:}
\begin{enumerate}
    \item On the right side panel of the screen, click on the \textbf{Packages} tab.
    \item Click on \textbf{Install}.
    \item Search for \textbf{stringr} and install.
\end{enumerate}
\textbf{Using a line of code:}
\begin{enumerate}
    \item \texttt{install.packages(packageName)}
    \item Make sure the package name is in quotation marks because it is a string.
\end{enumerate}

\subsection*{b. Tell R that you want to use this package in your script}
\begin{enumerate}
    \item Include \texttt{library(packageName)} at the top of your script file.
    \item The package name should \textbf{not} be in quotation marks.
\end{enumerate}

\section*{2) Initial Data Overview}
\begin{enumerate}
    \item Read the R help documentation about the function \texttt{read.csv}.
    \item Read in the CSV file \texttt{books.csv}.
    \item What are the column headers for this dataset?
    \item How many rows of data are in the dataset?
    \item How many unique authors are there in the dataset? (Use the \texttt{unique()} function.)
\end{enumerate}

\section*{3) Creating Data Subsets}
\begin{enumerate}
    \item Create a new data frame containing books with more than 45 pages.
    \item How many books have more than 45 pages?
    \item Create a new data frame containing books with more than 45 pages and written in English (but not Middle English).
    \begin{itemize}
        \item Read the R help for \texttt{str\_detect()}.
        \item Use \texttt{str\_detect()} to create a list of English language codes.
        \item Remove \texttt{"enm"} from your list.
        \item Create the subset using \texttt{subset()}.
    \end{itemize}
    \item Compute summary statistics for the average ratings of these books:
    \begin{enumerate}
        \item Minimum, maximum, range
        \item First quartile, median, third quartile
        \item Interquartile range, mean, variance
    \end{enumerate}
    \item Compute the correlation between the average rating and the number of pages. Interpret the result in 1-2 sentences.
\end{enumerate}

\section*{4) Data Visualization}
\textbf{All plots must have a descriptive title and appropriate axis labels.}
\begin{enumerate}
    \item Create a five-number summary (min, Q1, median, Q3, max) of the number of pages.
    \item Plot a histogram of the number of pages.
    \item Create a boxplot of the number of pages.
    \item Explain why these plots could be improved.
\end{enumerate}

\section*{5) Finding Percentiles}
\begin{enumerate}
    \item For each percentile from 95 to 99:
    \begin{itemize}
        \item Find the number of pages corresponding to that percentile using \texttt{quantile()}.
        \item Create a subset of books with fewer than that number of pages.
        \item Create a histogram of the number of pages.
    \end{itemize}
    \item Choose one percentile that is reasonable for subsetting the data and provides better graphs without losing too much data.
    \item For your selected percentile:
    \begin{itemize}
        \item Create a histogram and boxplot.
        \item Compute a five-number summary.
        \item Compare the new summary with the original summary.
    \end{itemize}
\end{enumerate}

\section*{6) QQ Plot}
\begin{enumerate}
    \item Create a QQ plot for the number of ratings (\texttt{ratings\_count} column) in English books with more than 45 pages.
    \begin{itemize}
        \item Read the help documentation for \texttt{qqnorm}.
        \item Generate the QQ plot.
        \item Add the reference line (\texttt{qqline}).
    \end{itemize}
    \item Based on your plot, is the number of ratings normally distributed? Explain briefly.
\end{enumerate}
\newpage
\section*{Grading Overview}

\textbf{Code File: 30 points}
\begin{itemize}
    \item Code file runs without errors: 10 points
    \item Code file is commented: 5 points
    \item Code file is complete: 15 points
\end{itemize}

\textbf{Report: 60 points}
\begin{itemize}
    \item Report formatting and completeness: 10 points
    \item Report accuracy/correctness: 50 points
\end{itemize}

\textbf{Note File: 10 points}
\begin{itemize}
    \item Note file has been updated with new material from this lab.
    \item Note file reflects student's own notes.
\end{itemize}

\end{document}
