\documentclass[12pt]{article}
\usepackage{amsmath}
\usepackage{listings}
\usepackage{xcolor}

\lstset{
    basicstyle=\ttfamily\small,
    keywordstyle=\color{blue}\bfseries,
    commentstyle=\color{green!50!black},
    stringstyle=\color{orange},
    numbers=left,
    numberstyle=\tiny,
    stepnumber=1,
    numbersep=5pt,
    showspaces=false,
    showstringspaces=false,
    frame=single,
    breaklines=true,
    postbreak=\mbox{\textcolor{red}{$\hookrightarrow$}\space}
}

\title{Lab 3 - Solution Code}
\author{}
\date{}

\begin{document}
\maketitle

\section*{Problem 1: Dataset Overview}
\begin{lstlisting}[language=R]
data(quakes) #1a
\end{lstlisting}
\textbf{Answer:} Load the \texttt{quakes} dataset.

\begin{lstlisting}[language=R]
head(quakes) #1b
\end{lstlisting}
\textbf{Answer:} Display the first few rows of the dataset.

\begin{lstlisting}[language=R]
dim(quakes) #1c   or  #nrow(quakes)
\end{lstlisting}
\textbf{Answer:} Display the dimensions (rows and columns) of the dataset.

\section*{Problem 2: Summary Statistics}
\begin{lstlisting}[language=R]
s <- summary(quakes$mag) #2a
s

range(quakes$mag)  #2a
\end{lstlisting}
\textbf{Answer:} Calculate the summary statistics and range of the \texttt{mag} column.

\begin{lstlisting}[language=R]
IQR <- s[5] - s[2] #2a
IQR

var(quakes$mag)
sd <- sd(quakes$mag)
sd

mean <- mean(quakes$mag)
mean

cv <- (sd / mean) * 100 #2a
cv
\end{lstlisting}
\textbf{Answer:} Compute the interquartile range, variance, standard deviation, mean, and coefficient of variation (CV).

\section*{Problem 3: Data Visualization}
\begin{lstlisting}[language=R]
hist(quakes$depth, 
     labels = TRUE, 
     main = "Depths", 
     xlab = 'depth') #3a

hist(quakes$mag, 
     labels = TRUE, 
     main = "Magnitudes", 
     xlab = 'mag') #3b

boxplot(quakes$stations, 
        labels = TRUE, 
        main = "Number of Stations", 
        ylab = "stations") #3c
\end{lstlisting}
\textbf{Answer:} Create histograms for \texttt{depth} and \texttt{mag}, and a boxplot for \texttt{stations}.

\section*{Problem 4:  Manipulating Data and Creating New Vectors}
\begin{lstlisting}[language=R]
depthMiles <- quakes$depth * 0.621371  #4a

under40stations <- quakes[quakes$stations < 40,] #4b
under40stations

over40stations <- quakes[quakes$stations >= 40,] #4b
over40stations
\end{lstlisting}
\textbf{Answer:}
\begin{itemize}
    \item Convert \texttt{depth} from kilometers to miles.
    \item Subset data for earthquakes recorded by fewer than 40 stations and those recorded by at least 40 stations.
\end{itemize}

\section*{Problem 5:  Summary Stats for Data Subsets}
\begin{lstlisting}[language=R]
dim(under40stations) #5a
dim(over40stations)  #5b

summary(depthMiles) #5ci
summary(quakes$depth)

summary(under40stations$depth) #5cii
summary(over40stations$depth) #5ciii
\end{lstlisting}
\textbf{Answer:}
\begin{itemize}
    \item Display the dimensions of the subsets.
    \item Summarize \texttt{depth} in miles and for both subsets.
\end{itemize}

\begin{lstlisting}[language=R]
mean(quakes$depth) #5d
var(quakes$depth)  #5d
summary(quakes$depth)[5] - summary(quakes$depth)[2] #5d

mean(under40stations$depth) #5d
var(under40stations$depth)  #5d
summary(under40stations$depth)[5] - summary(under40stations$depth)[2] #5d

mean(over40stations$depth) #5d
var(over40stations$depth)  #5d
summary(over40stations$depth)[5] - summary(over40stations$depth)[2] #5d
\end{lstlisting}
\textbf{Answer:} Compute the mean, variance, and IQR for the \texttt{depth} column in the full dataset and subsets.

\section*{Problem 6:  Data Visualization for Data Subsets}
\begin{lstlisting}[language=R]
par(mfrow = c(1, 2)) #6a

hist(under40stations$mag, #6b
     main = "Magnitudes < 40", 
     xlab = 'mag')

hist(over40stations$mag, #6b
     main = "Magnitudes >= 40", 
     xlab = 'mag')

boxplot(under40stations$depth, #6c
        ylim = c(0, max(quakes$depth)),
        main = "Depth < 40", 
        ylab = "depth")

boxplot(over40stations$depth, #6c
        ylim = c(0, max(quakes$depth)),
        main = "Depth >= 40", 
        ylab = "depth")
\end{lstlisting}
\textbf{Answer:}
\begin{itemize}
    \item Use \texttt{par()} to create side-by-side plots.
    \item Create histograms and boxplots for the subsets.
\end{itemize}

\section*{Problem 7: Order Statistics}
\begin{lstlisting}[language=R]
neww <- sort(over40stations$long) #7a
neww[18] #7b

xy_function <- function(x, y) {
z=x*y
  return(z)
}
result <- xy_function(3, 5)
print(result)  #7c
\end{lstlisting}

\textbf{Answer:}
\begin{itemize}
    \item Sort the \texttt{long} column for earthquakes recorded by at least 40 stations.
    \item Find the 18th order statistic.
    \item Find x*y function.
\end{itemize}

\end{document}
