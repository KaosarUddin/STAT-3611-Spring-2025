\documentclass{article}
\usepackage{graphicx}
\usepackage{booktabs}
\usepackage{hyperref}
\usepackage{float}

\title{STAT 3611 - Lab 14 Report}
\author{Student Name}
\date{April 28, 2025}

\begin{document}

\maketitle

\section{Introduction}
This report explores one-way ANOVA analysis using the \texttt{coagulation} and \texttt{penguins} datasets in R. We assess assumptions, generate visualizations, and evaluate group means using ANOVA.

\section{One-Way ANOVA: Coagulation Dataset}

\subsection{Data Summary and Assumptions}
\begin{itemize}
    \item Column names: [Insert column names from \texttt{coagulation} dataset].
    \item Total number of rows: [Insert number].
    \item Diet-wise data frame sizes:
    \begin{itemize}
        \item Diet A: [Insert count]
        \item Diet B: [Insert count]
        \item Diet C: [Insert count]
        \item Diet D: [Insert count]
    \end{itemize}
    \item Sample variances by diet: [Insert values and commentary on homogeneity assumption.]
\end{itemize}

\subsection{Boxplots of Coagulation Time by Diet}
\begin{figure}[H]
    \centering
    \includegraphics[width=0.85\textwidth]{coagulation_boxplot.png} % Replace with actual plot file
    \caption{Boxplots of Coagulation Time for Each Diet}
    \label{fig:coag_box}
\end{figure}

\subsection{Normality Check: QQ Plots for Each Diet}
\begin{figure}[H]
    \centering
    \includegraphics[width=0.85\textwidth]{qqplots_coagulation.png} % Replace with actual plot file
    \caption{QQ Plots of Coagulation Time by Diet}
    \label{fig:coag_qq}
\end{figure}

\subsection{ANOVA Test}
\begin{itemize}
    \item \textbf{H\textsubscript{0}}:....
    \item \textbf{H\textsubscript{A}}:....
    \item Summary of \texttt{aov()} and \texttt{summary()} output:
    \begin{itemize}
        \item Degrees of freedom: [Insert df]
        \item Sum of squares: [Insert values]
        \item p-value: [Insert p-value]
        \item Conclusion: [Reject/Fail to reject H\textsubscript{0} with explanation.]
    \end{itemize}
\end{itemize}

\section{One-Way ANOVA: Penguins Dataset}

\subsection{Data Summary and Assumptions}
\begin{itemize}
    \item Selected variables: \texttt{species} and \texttt{bill\_depth\_mm}
    \item NA rows removed using \texttt{na.omit()}
    \item Sample sizes by species:
    \begin{itemize}
        \item Adelie: [Insert count]
        \item Chinstrap: [Insert count]
        \item Gentoo: [Insert count]
    \end{itemize}
    \item Sample variances: [Insert variances and discuss common variance assumption.]
\end{itemize}

\subsection{Boxplots of Bill Depth by Species}
\begin{figure}[H]
    \centering
    \includegraphics[width=0.85\textwidth]{penguins_boxplot.png} % Replace with actual plot file
    \caption{Boxplots of Bill Depth by Penguin Species}
    \label{fig:peng_box}
\end{figure}

\subsection{Normality Check: Histograms and QQ Plots}
\begin{figure}[H]
    \centering
    \includegraphics[width=0.95\textwidth]{penguins_normality.png} % Replace with actual plot file
    \caption{Histograms and QQ Plots for Bill Depth by Species}
    \label{fig:peng_norm}
\end{figure}

\subsection{ANOVA Test}
\begin{itemize}
    \item \textbf{H\textsubscript{0}}:...
    \item \textbf{H\textsubscript{A}}:....
    \item Summary of \texttt{aov()} and \texttt{summary()} output:
    \begin{itemize}
        \item Degrees of freedom: [Insert df]
        \item Sum of squares: [Insert values]
        \item p-value: [Insert p-value]
        \item Conclusion: [Reject/Fail to reject H\textsubscript{0} with explanation.]
    \end{itemize}
\end{itemize}

\section{Conclusion}
Summarize key findings and discuss whether assumptions were met and how the ANOVA tests helped in comparing group means.

\section{References}
\begin{itemize}
    \item ....
    \item ....
\end{itemize}

\end{document}
