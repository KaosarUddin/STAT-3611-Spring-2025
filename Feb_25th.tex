

\documentclass{beamer}
\usepackage{graphicx}
\usepackage{amsmath}
\usepackage{xcolor}

\usetheme{Madrid}

\title{Week 6 :Confidence Intervals for the Mean}
\author{Kaosar}
\institute{Auburn University}
\date{Spring 2025}



\begin{document}

\frame{\titlepage}

\begin{frame}{Admin Stuff}
    \begin{itemize}
        \item Previous lab: closed-resources lab
        \item Note Sheet reminder
        \item One closed-resources lab remaining
    \end{itemize}
\end{frame}

\begin{frame}{Steps 1-2: Same as previous labs}
    \textbf{Steps 1-2: Same as previous labs}
    \begin{itemize}
        \item Load data: \texttt{data("dataset\_name")}
        \item Identify column headers: \texttt{head(dataset\_name)}
        \item Number of rows: \texttt{nrow(dataset\_name)}
        \item Compute statistics (mean, variance, standard deviation, etc.)
    \end{itemize}
    \textbf{Note:}
    \begin{itemize}
        \item R computes sample variance and sample standard deviation.
        \item We need a function to compute population variance and standard deviation.
    \end{itemize}
\end{frame}


\begin{frame}{Step 3: Sampling (Part 1)}
    \textbf{Step 3: Sampling}
    \begin{itemize}
        \item \textcolor{green}{\textbf{Generate sample of size} \underline{$n$}}
        \begin{itemize}
            \item \textit{Function to use:} \textcolor{red}{\texttt{sample(x, n)}}
            \item \textbf{x}: elements from which to choose; e.g., if you want to sample from column A in a data frame called “data”, then you can use \textbf{\underline{data\$A}}
            \item \textbf{n}: non-negative integer, giving the number of items to choose
        \end{itemize}
    \end{itemize}
\end{frame}

\begin{frame}{Step 3: Sampling (Part 2)}
    \begin{itemize}
        \item \textcolor{green}{\textbf{Produce} \underline{$m$} \textbf{samples like this one}}
        \begin{itemize}
            \item \textit{Function to use:} \textcolor{red}{\texttt{replicate(m, y)}}
            \item \textbf{m}: number of times to repeatedly evaluate some expression
            \item \textbf{y}: expression to be evaluated
        \end{itemize}
        \item \textcolor{green}{\textbf{Store output as a data frame}}
        \begin{itemize}
            \item \textit{Function to use:} \textcolor{red}{\texttt{data.frame(...)}}
        \end{itemize}
    \end{itemize}
\end{frame}

\begin{frame}{Step 4: Analyze the samples}
    \textbf{Step 4: Analyze the samples}
    \begin{itemize}
        \item \textcolor{green}{\textbf{Create vector}}  
        \textit{Function to use:} \textcolor{red}{\texttt{c()}}

        \item \textcolor{green}{\textbf{Iterate through the samples}}  
        \textcolor{red}{\texttt{for(i in 1:\underline{m}) \{} \\  
        \quad \# compute what you need \\  
        \texttt{\}}}

        \item \textbf{\underline{m}}: number of samples

        \item \textcolor{green}{\textbf{Compute sample mean, sample variance, sample standard deviation}}  
        \textit{Function to use:} \textcolor{red}{\texttt{mean()} \quad \texttt{var()} \quad \texttt{sd()}}
    \end{itemize}
\end{frame}

\begin{frame}{Step 5: Distribution Quantiles}
    \textit{Note: Different ways to approach confidence interval questions.}

    \textbf{Step 5: Distribution Quantiles}
    \begin{itemize}
        \item \textcolor{green}{\textbf{z-value that corresponds to a 2 sided (1-alpha)\% confidence interval}}  
        
        \quad You can use: \textcolor{red}{\texttt{qnorm(1-(alpha/2))}}
        
        \item \textcolor{green}{\textbf{t-value that corresponds to a 2 sided (1-alpha)\% confidence interval with n = 10}}  
        
        \quad You can use: \textcolor{red}{\texttt{qt(1-(alpha/2), \underline{n-1})}}  
        
        \quad \textbf{\underline{n-1}}: degree of freedom for the t distribution
    \end{itemize}
\end{frame}
\begin{frame}{Step 6: Confidence Intervals Using Duration Samples}
    \textbf{Step 6: Confidence Intervals Using Duration Samples}
    \begin{itemize}
        \item \textcolor{green}{\textbf{Create vector}}  
        \textit{Function to use:} \textcolor{red}{\texttt{c()}}

        \item \textcolor{green}{\textbf{Perform test}}  
        \textit{Function to use:} \textcolor{red}{\texttt{t.test(x, ...)}}

        \item \textcolor{green}{\textbf{Get interval bounds}}  
        \[
        \text{sample\_means} - t \times \frac{\text{sample\_sds}}{\sqrt{10}}
        \]

        \item \textcolor{green}{\textbf{Verify if the true mean is in the interval.}}
    \end{itemize}
\end{frame}

\begin{frame}{Step 7a: Data Frame Modification}
    \textbf{Step 7a: Data Frame Modification}
    \begin{itemize}
        \item \textcolor{green}{\textbf{Append vectors}}  
        \textit{Function to use:} \textcolor{red}{\texttt{sample <- rbind(sample, sample\_means)}}
    \end{itemize}
\end{frame}

\begin{frame}{Step 7b: Data Frame Modification}
    \textbf{Step 7b: Data Frame Modification}
    \begin{itemize}
        \item \textcolor{green}{\textbf{Rename rows}}  
        \textit{Function to use:} \textcolor{red}{\texttt{row.names(samples) <- c("s1", ..., "s10", "x-bar", "ci\_lower\_t", ...)}}
    \end{itemize}
\end{frame}

\begin{frame}{Step 8: Write a Data Frame to a CSV File}
    \textbf{Step 8: Write a Data Frame to a CSV File}
    \begin{itemize}
        \item \textcolor{green}{\textbf{Function to use:}}  
        \textcolor{red}{\texttt{write.csv()}}
    \end{itemize}
\end{frame}


\begin{frame}{Lab Instructions}
    \textbf{R Script}
    \begin{itemize}
        \item Must be commented.
        \item The script you submit must run without errors and produce the correct output.
    \end{itemize}
    \textbf{Report}
    \begin{itemize}
        \item Template available on Canvas.
        \item Some sections may require additional descriptions/summaries.
    \end{itemize}
    \textbf{Additional Items}
    \begin{itemize}
        \item Updated Note Sheet
        \item The CSV file
    \end{itemize}
\end{frame}

\end{document}
