\documentclass{beamer}
\usepackage{graphicx}
\usepackage{xcolor}

\usetheme{Madrid}

\title{Lab 7}
\author{STAT 3611: Probability \& Statistics II Lab}
\date{}

\begin{document}

% Slide 1: Title
\begin{frame}
    \titlepage
\end{frame}

% Slide 2: Admin Stuff
\begin{frame}{Admin Stuff}
    \begin{itemize}
        \item Attendance is required!
    \end{itemize}
\end{frame}

% Slide 3: Admin Stuff
\begin{frame}{Admin Stuff}
    \begin{itemize}
        \item Your code should align with your report and uploaded documents.
    \end{itemize}
\end{frame}

% Slide 4: Admin Stuff
\begin{frame}{Admin Stuff}
    \begin{itemize}
        \item You should be using your note sheet if you get stuck during an open resource lab.
        \item Lab 10 (April 2) will be the second restricted reference lab.
        \item Make sure you are comfortable using the built-in R help and that your note sheet has been updated.
        \item Make sure your FULL note sheet is being submitted on Canvas each week; this should have all of your notes from all of the labs so far.
    \end{itemize}
\end{frame}

% Slide 5: Adding to Vectors Using c()
\begin{frame}{Adding to Vectors Using \texttt{c()}}
    \begin{itemize}
        \item Make the original list, maybe with data in it:
        \begin{itemize}
            \item \texttt{friends <- c("Jason")}
        \end{itemize}
        \item Add a new entry to the vector:
        \begin{itemize}
            \item \texttt{friends <- c(friends, "Kaley")}
        \end{itemize}
    \end{itemize}
\end{frame}

% Slide 6: For Loops
\begin{frame}{For Loops}
    \begin{itemize}
        \item Execute a block of code a fixed number of times.
    \end{itemize}

    \textbf{Example:}
    \begin{block}{R Code}
for (x in 1:10) {
    print(x + 2)
}
    \end{block}
    \begin{itemize}
        \item Loops must start and end with \texttt{\{\}}.
    \end{itemize}
\end{frame}

% Slide 7: Manipulating Vector Values
\begin{frame}{Manipulating Vector Values}
    \textbf{Multiply All Values by a Constant}
    \begin{itemize}
        \item \texttt{myVector <- c(1,2,3)}
        \item \texttt{myVector3x <- 3 * myVector}
    \end{itemize}

    \textbf{Add a Constant to Each Value}
    \begin{itemize}
        \item \texttt{myVector <- c(4,5,6)}
        \item \texttt{myVectorPlus2 <- 2 + myVector}
    \end{itemize}

    \textbf{Add Vectors of the Same Length}
    \begin{itemize}
        \item \texttt{myVector1 <- c(1,2,3)}
        \item \texttt{myVector2 <- c(4,5,6)}
        \item \texttt{myVectorSum <- myVector1 + myVector2}
    \end{itemize}
\end{frame}

% Slide 8: Confidence Intervals - Variance
\begin{frame}{Confidence Intervals: Variance}
    \begin{itemize}
        \item The reference distribution is the Chi-Squared distribution.
        \item Like the t-distribution, we have to include the degrees of freedom.
        \item Remember that \texttt{S} is the sample standard deviation.
        \item Use the \texttt{qchisq()} function in R.
    \end{itemize}
\end{frame}

% Slide 9: Confidence Intervals - P-Hat (Population Proportion)
\begin{frame}{Confidence Intervals: P-Hat (Population Proportion)}
    \begin{itemize}
        \item We conduct a simple random sample of 90 people in this university and find that 74 of them identify as a student.
    \end{itemize}

    \textbf{Example Calculation:}
    \begin{block}{R Code}
p_hat <- 74 / 90
# Output: 0.82
    \end{block}
    
    \begin{itemize}
        \item Sample size: \texttt{n = 90}
    \end{itemize}
\end{frame}

% Slide 10: Distribution Quantiles
\begin{frame}{Distribution Quantiles}
    \begin{itemize}
        \item Functions in R are available to get the quantiles from a variety of distributions:
        \begin{itemize}
            \item \texttt{qnorm()} - Normal distribution
            \item \texttt{qt()} - t-distribution
            \item \texttt{qchisq()} - Chi-squared distribution
        \end{itemize}
    \end{itemize}
\end{frame}

% Slide 11: Lab Submissions
\begin{frame}{Lab Submissions}
    \begin{itemize}
        \item \textbf{R script:}
        \begin{itemize}
            \item Must be commented.
            \item If we run the script you turn in, it must run without errors and give the same output that you have in your report.
        \end{itemize}

        \item \textbf{Report:}
        \begin{itemize}
            \item Template available on Canvas.
            \item There are some places where you will need to add descriptions/summaries.
        \end{itemize}

        \item \textbf{CSV File:} Required for lab submissions.

        \item \textbf{Updated Note Sheet:} Must be submitted weekly, including notes from all previous labs.
    \end{itemize}
\end{frame}

\end{document}
