\documentclass[12pt]{article}
\usepackage{amsmath}
\usepackage{listings}
\usepackage{xcolor}

\lstset{
    basicstyle=\ttfamily\small,
    keywordstyle=\color{blue}\bfseries,
    commentstyle=\color{green!50!black},
    stringstyle=\color{orange},
    numbers=left,
    numberstyle=\tiny,
    stepnumber=1,
    numbersep=5pt,
    showspaces=false,
    showstringspaces=false,
    frame=single,
    breaklines=true,
    postbreak=\mbox{\textcolor{red}{$\hookrightarrow$}\space}
}

\title{Lab 4 - Solution Code}
\author{}
\date{}

\begin{document}
\maketitle

\section*{Problem 1: Dataset Overview}
\begin{lstlisting}[language=R]
library(stringr)
books <- read.csv("books.csv") #2b

dim(books) #2d

uni_aut <- unique(books$authors) #2e
length(uni_aut) #2e
\end{lstlisting}
\textbf{Answer:} Load and examine the dataset.

\section*{Problem 2: Data Filtering and Processing}
\begin{lstlisting}[language=R]
more_45_page <- books[books$num_pages > 45,] #3a
dim(more_45_page) #3b

uni_Lang_more_45_page <- unique(more_45_page$language_code)
uni_Lang_more_45_page_en <- uni_Lang_more_45_page[str_detect(uni_Lang_more_45_page, "en")] #3c_ii
uni_Lang_more_45_page_en_not_enm <- uni_Lang_more_45_page_en[uni_Lang_more_45_page_en != "enm"] #3c_iii
Eng_more_45 <- subset(more_45_page, more_45_page$language_code %in% uni_Lang_more_45_page_en_not_enm) #3c_iv
dim(Eng_more_45)
unique(Eng_more_45$language_code) 
\end{lstlisting}
\textbf{Answer:} Filter books with more than 45 pages and in English.

\section*{Problem 3: Summary Statistics}
\begin{lstlisting}[language=R]
s <- summary(Eng_more_45$average_rating)
s
range(Eng_more_45$average_rating)
mean(Eng_more_45$average_rating)
var(Eng_more_45$average_rating)
IQR <- s[5] - s[2]
IQR
cor(Eng_more_45$average_rating, Eng_more_45$num_pages)
\end{lstlisting}
\textbf{Answer:} Compute key summary statistics.

\section*{Problem 4: Data Visualization}
\begin{lstlisting}[language=R]
summary(Eng_more_45$num_pages) #4a
par(mfrow = c(1,2))
hist(Eng_more_45$num_pages, main = "Histogram of Page Numbers", xlab = "Number of Pages") #4b
boxplot(Eng_more_45$num_pages, main = "Boxplot of Page Numbers", xlab = "Number of Pages") #4c
\end{lstlisting}
\textbf{Answer:} Visualize data distributions.

\section*{Problem 5: Percentile Analysis}
\begin{lstlisting}[language=R]
#5a 
#95th percentile of page length 
quan95 <- quantile(Eng_more_45$num_pages, probs = 0.95) 
quan95 
fewerquan95 <- subset(Eng_more_45, Eng_more_45$num_pages < quan95) 
fewerquan95 
hist(fewerquan95$num_pages) 
#96th percentile of page length 
quan96 <- quantile(Eng_more_45$num_pages, probs = 0.96) 
fewerquan96 <- subset(Eng_more_45, Eng_more_45$num_pages < quan96) 
fewerquan96 
hist(fewerquan96$num_pages) 
#97th percentile of page length 
quan97 <- quantile(Eng_more_45$num_pages, probs = 0.97) 
fewerquan97 <- subset(Eng_more_45, Eng_more_45$num_pages < quan97) 
fewerquan97 
hist(fewerquan97$num_pages) 
#98th percentile of page length 
quan98 <- quantile(Eng_more_45$num_pages, probs = 0.98) 
fewerquan98 <- subset(Eng_more_45, Eng_more_45$num_pages < quan98) 
fewerquan98 
hist(fewerquan98$num_pages) 
#99th percentile of page length 
quan99 <- quantile(Eng_more_45$num_pages, probs = 0.99) 
quan99 
fewerquan99 <- subset(Eng_more_45, Eng_more_45$num_pages < quan99) 
fewerquan99 
hist(fewerquan99$num_pages) 
#or 
par(mfrow = c(5,1), mar=c(2,4, 2, 2)) 
percentiles <- c(0.95, 0.96, 0.97, 0.98, 0.99) 
for (q in percentiles){ 
currentQuantile <- quantile(Eng_more_45$num_pages, probs = q) 
currentSubset <- Eng_more_45[Eng_more_45$num_pages < currentQuantile, ] 
t
 itle <- paste(as.character(q*100)," percentile num pages") 
hist(currentSubset$num_pages, main = title) 
} 
#or 
par(mfrow = c(5,1)) 
for (q in 95:99){ 
currentQuantile <- quantile(Eng_more_45$num_pages, probs = q/100) 
currentSubset <- Eng_more_45[Eng_more_45$num_pages < currentQuantile, ] 
title <- paste(as.character(q)," percentile num pages") 
hist(currentSubset$num_pages, main = title) 
}

for (q in 95:99){ 
currentQuantile <- quantile(Eng_more_45$num_pages, probs = q/100) 
print(currentQuantile) 
} 
#5b 
#Based on the following code, with 99th percentile, we lose less data, so we should select 99th 
for (q in 95:99){  
currentQuantile <- quantile(Eng_more_45$num_pages, probs = q/100) 
currentSubset <- Eng_more_45[Eng_more_45$num_pages < currentQuantile, ] 
print(dim(currentSubset)) 
} 
#5c 
par(mfrow = c(1,2))  
hist(fewerquan99$num_pages,  
main = "The histogram of #99th percentile", 
xlab = "Number of pages" 
) 
boxplot(fewerquan99$num_pages,  
main = "The boxplot of #99th percentile", 
xlab = "Number of pages" 
) 
summary(fewerquan99$num_pages) 
\end{lstlisting}
\textbf{Answer:}
\begin{itemize}
    \item Analyze the each percentile from 95 to 99 of page numbers.
    \item Choose one percentile that is reasonable for subsetting the data and provides better graphs without losing too much data.
    \item Generate a histogram, Boxplot, and summary  of the percentile.
\end{itemize}


\section*{Problem 6: Normality Check}
\begin{lstlisting}[language=R]
par(mfrow = c(1,1))
qqnorm(Eng_more_45$ratings_count)
qqline(Eng_more_45$ratings_count)
\end{lstlisting}
\textbf{Answer:} Generate a Q-Q plot to check normality.

\end{document}
