\documentclass{article}
\usepackage{graphicx}
\usepackage{booktabs}
\usepackage{hyperref}
\usepackage{float}

\title{STAT 3611 - Lab 11 Report}
\author{Your First and Last Name}
\date{Submission Date}

\begin{document}

\maketitle

\section{Introduction}
This report presents the analysis for Lab 11, which includes testing for the median and variance using nonparametric methods and simple linear regression analysis using the \texttt{islands} and \texttt{starwars} datasets.

\section{Analysis Using the islands Dataset}

\subsection{Initial Data Overview}
\begin{itemize}
    \item The \texttt{islands} dataset was loaded.
    \item Units are in 10,000s of square miles.
\end{itemize}

\subsection{Normality Check}
\begin{itemize}
    \item QQ plot created with reference line.
    \begin{figure}[H]
    \centering
    \includegraphics[width=0.8\textwidth]{qq_plots.png} % update this with actual filename
    \caption{ QQ plots .}
    \label{fig:height_dists}
\end{figure}
    \item Conclusion: \textit{(State if distribution appears normal or skewed, and why)}
\end{itemize}

\subsection{Sampling and Visualization}
\begin{itemize}
    \item Five random samples of size 15 created.
    \item Combined into a single dataframe with sample indices.
\end{itemize}

\begin{figure}[H]
    \centering
    \includegraphics[width=0.8\textwidth]{boxplot_samples.png} % update this with actual filename
    \caption{Boxplots of five samples from the islands dataset.}
    \label{fig:islands_boxplot}
\end{figure}

\subsection{Sign Tests for Median = 100}
\begin{itemize}
    \item Each sample tested using \texttt{binom.test()}.
    \item Significance level: 0.05
    \item Summary of results:
    \begin{itemize}
        \item Sample 1: \_\_
        \item Sample 2: \_\_
        \item Sample 3: \_\_
        \item Sample 4: \_\_
        \item Sample 5: \_\_
    \end{itemize}
    \item Observations: \textit{(Brief explanation on variability across tests)}
\end{itemize}

\subsection{Signed Rank Tests for Median = 100}
\begin{itemize}
    \item Each sample tested using \texttt{wilcox.test()}.
    \item Significance level: 0.05
    \item Summary of results:
    \begin{itemize}
        \item Sample 1: \_\_
        \item Sample 2: \_\_
        \item Sample 3: \_\_
        \item Sample 4: \_\_
        \item Sample 5: \_\_
    \end{itemize}
    \item Comparison to sign tests: \textit{(Brief explanation)}
    \item Preferred method: \textit{(State which test is better and why)}
\end{itemize}

\section{Analysis Using the starwars Dataset}

\subsection{Initial Cleaning and Overview}
\begin{itemize}
    \item The \texttt{starwars} dataset was loaded.
    \item A dataframe with height and species was created.
    \item Missing values were removed using \texttt{na.omit()}.
    \item Number of rows removed: \_\_
\end{itemize}

\subsection{Group Separation}
\begin{itemize}
    \item Human heights extracted.
    \item Alien heights extracted (non-human, non-droid).
\end{itemize}

\subsection{Visualization of Heights}
\begin{figure}[H]
    \centering
    \includegraphics[width=0.8\textwidth]{heights_plots.png} % update this with actual filename
    \caption{Histograms and QQ plots of human and alien heights.}
    \label{fig:height_dists}
\end{figure}

\textbf{Observation:} \textit{(Describe distribution shape and differences)}

\subsection{Descriptive Statistics}
\begin{itemize}
    \item Humans: mean = \_\_, median = \_\_, sd = \_\_
    \item Aliens: mean = \_\_, median = \_\_, sd = \_\_
\end{itemize}

\subsection{Wilcoxon Test for Median Difference}
\begin{itemize}
    \item Null: 
    \item Alt:
    \item Significance level: 0.02
    \item Result: \_\_, Conclusion: \textit{(Interpretation)}
\end{itemize}

\subsection{Test for Variance}
\begin{itemize}
    \item Null: variances are equal
    \item Alt: variances are different
    \item Method: \texttt{var.test()}
    \item Significance level: 0.02
    \item Result: \_\_, Conclusion: \textit{(Interpretation)}
\end{itemize}

\section{Linear Regression: Height vs Mass}
\begin{itemize}
    \item Scatter plot of mass (y) vs height (x) created.
    \begin{figure}[H]
    \centering
    \includegraphics[width=0.8\textwidth]{Scatter_plot.png} % update this with actual filename
    \caption{Scatter plot of mass (y) vs height (x) created .}
    \label{fig:Scatter_plot}
\end{figure}
    \item Linear model fit using \texttt{lm()}.
    \item Regression line added using \texttt{abline()}.
    \item Regression equation: \(\hat{y} = b_0 + b_1x\)
    \item Slope interpretation: \textit{(Explain meaning in context)}
    \item \(R^2\) value: \_\_, interpretation: \textit{(Explanation)}
    \item Model fit: Appears reasonable/unreasonable \textit{(Justify)}
\end{itemize}

\section{Conclusion}
Summarize findings across all sections, comparing effectiveness of median tests, insights from visualization, and conclusions from regression and variance analysis.

\section{References}
List any documentation used (e.g., R help files, packages: ggpubr, dplyr, MASS).

\end{document}