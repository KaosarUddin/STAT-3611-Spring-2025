\documentclass{article}
\usepackage{geometry}
\geometry{a4paper, margin=1in}
\usepackage{hyperref}
\usepackage{fancyhdr}
\usepackage{listings}
\usepackage{xcolor}
\usepackage{tcolorbox}
\pagestyle{fancy}
\usepackage{enumitem}
\usepackage{graphicx}

\fancyhead[L]{STAT 3611 Assignment (Lab 6)}
\fancyhead[R]{March 03 at 11:59pm}

\lstset{
  language=R,
  basicstyle=\ttfamily\footnotesize,
  keywordstyle=\color{blue},
  commentstyle=\color{green},
  stringstyle=\color{red},
  numbers=left,
  numberstyle=\tiny,
  stepnumber=1,
  breaklines=true,
  showstringspaces=false
}

\begin{document}

\title{STAT 3611 Lab 6 Solution Code}
\author{Auburn University}
\date{Spring 2025}
\maketitle

\section{Problem 3.1: Initial Data Overview}
\begin{tcolorbox}
\begin{lstlisting}
# Load the dataset
data("faithful")
colnames(faithful) # Column headers
nrow(faithful) # Number of rows
\end{lstlisting}
\end{tcolorbox}
Answer: Load and examine the dataset.

\section{Problem 3.2: Summary Statistics}
\begin{tcolorbox}
\begin{lstlisting}
# Compute statistics for numerical columns
summary_stats <- data.frame(
  Mean = sapply(faithful, function(x) if(is.numeric(x)) mean(x, na.rm = TRUE) else NA),
  Variance = sapply(faithful, function(x) if(is.numeric(x)) var(x, na.rm = TRUE) else NA),
  Std_Dev = sapply(faithful, function(x) if(is.numeric(x)) sd(x, na.rm = TRUE) else NA),
  Coef_Var = sapply(faithful, function(x) if(is.numeric(x)) sd(x, na.rm = TRUE) / mean(x, na.rm = TRUE) else NA)
)
print(summary_stats)
\end{lstlisting}
\end{tcolorbox}
Answer: Compute key summary statistics, including the population coefficient of variation.

\section{Problem 3.3: Sampling}
\begin{tcolorbox}
\begin{lstlisting}
# Generate multiple samples and convert to a data frame
samples <- replicate(120, sample(faithful$eruptions, 15, replace=TRUE))
samples_df <- data.frame(samples)
print(samples_df)
\end{lstlisting}
\end{tcolorbox}
Answer: Create a data frame of 120 samples.
\section{Problem 3.4: Analyze the Samples}
\begin{tcolorbox}
\begin{lstlisting}
# Create empty vectors to store sample statistics
sample_means <- c()
sample_variances <- c()
sample_std_devs <- c()

# Generate 120 samples from the eruption duration column
samples <- replicate(120, sample(faithful$eruptions, 15, replace = TRUE), simplify = FALSE)

# Compute statistics for each sample
for (i in 1:120) {
  sample_means <- c(sample_means, mean(samples[[i]]))
  sample_variances <- c(sample_variances, var(samples[[i]]))
  sample_std_devs <- c(sample_std_devs, sd(samples[[i]]))
}

# Compute average and population variance of each new vector
avg_sample_means <- mean(sample_means)
avg_sample_vars <- mean(sample_variances)
avg_sample_sds <- mean(sample_std_devs)

pop_var_sample_means <- var(sample_means)
pop_var_sample_vars <- var(sample_variances)
pop_var_sample_sds <- var(sample_std_devs)

# Compute bias (true parameter - estimate)
true_mean <- mean(faithful$eruptions)
true_variance <- var(faithful$eruptions)
true_sd <- sd(faithful$eruptions)

bias_means <- true_mean - avg_sample_means
bias_variance <- true_variance - avg_sample_vars
bias_sd <- true_sd - avg_sample_sds

# Store results in a data frame
sample_analysis_results <- data.frame(
  Sample_ID = 1:120,
  Sample_Mean = sample_means,
  Sample_Variance = sample_variances,
  Sample_Std_Dev = sample_std_devs
)
print(sample_analysis_results)

# Display bias and population variance results
bias_results <- data.frame(
  Metric = c("Sample Means", "Sample Variances", "Sample Std Devs"),
  Bias = c(bias_means, bias_variance, bias_sd),
  Population_Variance = c(pop_var_sample_means, pop_var_sample_vars, pop_var_sample_sds)
)
print(bias_results)
\end{lstlisting}
\end{tcolorbox}
Answer: Compute the sample means, variances, and standard deviations, along with their averages, population variances, and bias calculations.

\section{Problem 3.5: Distribution Quantiles}
\begin{tcolorbox}
\begin{lstlisting}
# Compute z-value and t-value
alpha <- 0.05  # 95% confidence level
z_value <- qnorm(1 - (alpha / 2))
n <- 10  # Sample size
t_value <- qt(1 - (alpha / 2), df = n - 1)
cat("Z-value:", z_value, "\n")
cat("T-value for n =", n, ":", t_value, "\n")
\end{lstlisting}
\end{tcolorbox}
Answer: Find quantiles for normal and t-distributions.


\section{Problem 3.6: Confidence Intervals Using Duration Samples}
\begin{tcolorbox}
\begin{lstlisting}
# Compute confidence intervals
set.seed(123)
true_mean <- mean(faithful$eruptions)
samples <- replicate(120, sample(faithful$eruptions, 15, replace = TRUE), simplify = FALSE)
sample_sds <- c()
lower_bounds <- c()
upper_bounds <- c()
contains_true_mean <- c()

for (i in 1:120) {
  sample_sds <- c(sample_sds, sd(samples[[i]]))
  margin_of_error <- qt(0.975, df = 14) * (sample_sds[i] / sqrt(15))
  lower_bound <- mean(samples[[i]]) - margin_of_error
  upper_bound <- mean(samples[[i]]) + margin_of_error
  lower_bounds <- c(lower_bounds, lower_bound)
  upper_bounds <- c(upper_bounds, upper_bound)
  contains_true_mean <- c(contains_true_mean, (true_mean >= lower_bound & true_mean <= upper_bound))
}

# Store confidence interval results in a data frame
confidence_results <- data.frame(
  Sample_ID = 1:120,
  Sample_Mean = sapply(samples, mean),
  Sample_SD = sample_sds,
  Lower_Bound = lower_bounds,
  Upper_Bound = upper_bounds,
  Contains_True_Mean = contains_true_mean
)
print(confidence_results)

# Perform a t-test on a sample
sampled_values <- sample(faithful$eruptions, 15, replace = TRUE)
t.test(sampled_values)
\end{lstlisting}
\end{tcolorbox}
Answer: Compute confidence intervals for each of the 120 samples and check whether the true mean is contained within each interval.


\section{Problem 3.7: Data Frame Modification}
\begin{tcolorbox}
\begin{lstlisting}
# Append sample means and confidence intervals to the dataset
samples_matrix <- do.call(cbind, samples)
samples_df <- as.data.frame(samples_matrix)
samples_df <- rbind(samples_df, sample_means, lower_bounds, upper_bounds)
row.names(samples_df) <- c(paste0("s", 1:n), "x-bar", "ci_lower", "ci_upper")
print(samples_df)
\end{lstlisting}
\end{tcolorbox}
Answer: Append confidence interval data to the sample data frame.

\section{Problem 3.8: Writing to CSV}
\begin{tcolorbox}
\begin{lstlisting}
# Write the data frame to a CSV file
write.csv(samples_df, "samples_results.csv", row.names = TRUE)
\end{lstlisting}
\end{tcolorbox}
Answer: Save the results to a CSV file.

\end{document}
