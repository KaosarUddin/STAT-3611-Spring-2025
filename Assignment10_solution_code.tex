\documentclass{article}
\usepackage{listings}
\usepackage{xcolor}

\lstset{
  language=R,
  basicstyle=\ttfamily\small,
  keywordstyle=\color{blue},
  commentstyle=\color{green!60!black},
  stringstyle=\color{red},
  frame=single,
  breaklines=true,
  numbers=left,
  numberstyle=\tiny,
  captionpos=b
}

\begin{document}

\section*{STAT 3611 Lab 10 Solution Code}

\subsection*{1. Initial Data Overview of \texttt{birthwt}}
\begin{lstlisting}
# Load the required libraries
# install.packages("MASS")
library(MASS)

# Load birthwt dataset
data("birthwt")
colnames(birthwt)
nrow(birthwt)

# Add column for birthweight in pounds
birthwt$weight_lb <- birthwt$bwt / 454

# Create subsets for smokers and non-smokers
smokers <- subset(birthwt, smoke == 1)
nonsmokers <- subset(birthwt, smoke == 0)

# Summary statistics
library(e1071)

# Smokers
n_smokers <- nrow(smokers)
mean_smokers_grams <- mean(smokers$bwt)
var_smokers_grams <- var(smokers$bwt)
skew_smokers_grams <- skewness(smokers$bwt)

# Non-smokers
n_nonsmokers <- nrow(nonsmokers)
mean_nonsmokers_grams <- mean(nonsmokers$bwt)
var_nonsmokers_grams <- var(nonsmokers$bwt)
skew_nonsmokers_grams <- skewness(nonsmokers$bwt)
\end{lstlisting}
Answer:
- Column headers include \texttt{low, age, lwt, race, smoke, ptl, ht, ui, ftv, bwt}.
- Dataset contains \texttt{189 rows}.
- Calculated and compared summary statistics suggest the means of the two groups (smokers and non-smokers) will differ.

\subsection*{2. Two-Sample Hypothesis Test}
\begin{lstlisting}
# QQ plots for normality
par(mfrow=c(1,3))
qqnorm(birthwt$bwt, main="All Babies")
qqline(birthwt$bwt, col="blue")
qqnorm(smokers$bwt, main="Smokers")
qqline(smokers$bwt, col="red")
qqnorm(nonsmokers$bwt, main="Non-smokers")
qqline(nonsmokers$bwt, col="green")

# Two-sided t-test
# H0: mean_smokers = mean_nonsmokers
# Ha: mean_smokers != mean_nonsmokers
t.test(smokers$bwt, nonsmokers$bwt, alternative="two.sided")

# One-sided t-test
# H0: mean_smokers >= mean_nonsmokers
# Ha: mean_smokers < mean_nonsmokers
t.test(smokers$bwt, nonsmokers$bwt, alternative="less")
\end{lstlisting}
Answer:
- Two-sided test rejects the null hypothesis, suggesting significant difference in mean birth weights.
- One-sided test indicates smoking mothers have significantly lower birth weights than non-smoking mothers.

\subsection*{3. Hypothesis Tests for the Proportion Parameter}
\begin{lstlisting}
# Load precip dataset
data("precip")

# Sample 25 cities
set.seed(123)
sample_precip <- sample(precip, 25)

# Proportion test
# H0: p = 0.65
# Ha: p != 0.65
successes <- sum(sample_precip >= 20)
binom.test(successes, 25, p=0.65, conf.level=0.90)

# One-sided proportion test
# H0: p = 0.65
# Ha: p > 0.65
binom.test(successes, 25, p=0.65, alternative="greater", conf.level=0.90)

# True proportion in full dataset
true_proportion <- mean(precip >= 20)
true_proportion
\end{lstlisting}
Answer:
- Conducted two-sided and one-sided proportion tests for rainfall data.
- Comparison to the full dataset proportion suggests the tests provided accurate results. Any discrepancies could be due to sampling variability.

\end{document}
