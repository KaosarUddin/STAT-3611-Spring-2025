\documentclass{beamer}
\usepackage{graphicx}
\usepackage{amsmath}
\usepackage{xcolor}

\usetheme{Madrid}

\title{Week 6: Confidence intervals for standard deviation and proportion parameter}
\author{Kaosar}
\institute{Auburn University}
\date{Spring 2025}


\begin{document}

\frame{\titlepage}

% Slide 2: Admin Stuff
\begin{frame}{Admin Stuff}
    \begin{itemize}
        \item Your code should align with your report and uploaded documents.
        \item You should be using your note sheet if you get stuck during an open resource lab.
        \item Make sure you are comfortable using the built-in R help and that your note sheet has been updated.
        \item Make sure your FULL note sheet is being submitted on Canvas each week; this should have all of your notes from all of the labs so far.
    \end{itemize}
\end{frame}

% Slide 3: Adding to Vectors Using c()
\begin{frame}{Adding to Vectors Using \texttt{c()}}
    \begin{itemize}
        \item Create an initial vector:
        \begin{itemize}
            \item \texttt{friends <- c("Jason")}
        \end{itemize}
        \item Add a new entry to the vector:
        \begin{itemize}
            \item \texttt{friends <- c(friends, "Kaley")}
        \end{itemize}
    \end{itemize}
\end{frame}

% Slide 4: For Loops
\begin{frame}{For Loops}
    \begin{itemize}
        \item Execute a block of code a fixed number of times.
    \end{itemize}
    \begin{block}{Function Syntax}
\texttt{Loop\_functionName <- function(parameters) \{}
\begin{itemize}
    \item Function Body Code
    \item \texttt{print(returnContents)}
\end{itemize}
\texttt{\}}
    \end{block}

    \begin{itemize}
        \item Loops must start and end with \texttt{\{\}}.
    \end{itemize}
\end{frame}

% Slide 5: Manipulating Vector Values
\begin{frame}{Manipulating Vector Values}
    \textbf{Multiply All Values by a Constant}
    \begin{itemize}
        \item \texttt{myVector <- c(1,2,3)}
        \item \texttt{myVector3x <- 3 * myVector}
    \end{itemize}

    \textbf{Add a Constant to Each Value}
    \begin{itemize}
        \item \texttt{myVector <- c(4,5,6)}
        \item \texttt{myVectorPlus2 <- 2 + myVector}
    \end{itemize}

    \textbf{Add Vectors of the Same Length}
    \begin{itemize}
        \item \texttt{myVector1 <- c(1,2,3)}
        \item \texttt{myVector2 <- c(4,5,6)}
        \item \texttt{myVectorSum <- myVector1 + myVector2}
    \end{itemize}
\end{frame}

% Slide 6: Confidence Intervals - Variance
\begin{frame}{Confidence Intervals: Variance}
    \begin{itemize}
        \item The reference distribution is the \textbf{Chi-Squared Distribution}.
        \item Like the t-distribution, we have to include the \textbf{degrees of freedom}.
        \item Remember that \( S \) is the sample standard deviation.
    \end{itemize}

    \vspace{0.5cm}
    \textbf{Chi-Squared Confidence Interval Formula:}

    \[
    \frac{(n - 1) S^2}{\chi^2_{\alpha/2, n-1}} \leq \sigma^2 \leq \frac{(n - 1) S^2}{\chi^2_{1-\alpha/2, n-1}}
    \]

    \vspace{0.3cm}
    \textbf{Using R:} The function \texttt{qchisq()} can be used to find critical values.

    \begin{block}{R Code}
qchisq(p, df)  # Gets quantiles of Chi-Squared distribution
    \end{block}
\end{frame}

\begin{frame}{Confidence Intervals: P-Hat (Population Proportion)}
    \begin{block}{Formula for Confidence Interval}
        \[
        \hat{p} \pm z^* \times \sqrt{\frac{\text{sample proportion} \times (1 - \text{sample proportion})}{n}}
        \]
    \end{block}

    \textbf{Example:}
    \begin{itemize}
        \item We conduct a simple random sample of 90 people in this university.
        \item 74 of them identify as a student.
    \end{itemize}

    \textbf{Calculation:}
    \[
    \hat{p} = \frac{74}{90} = 0.82 \quad \quad \mathbf{n = 90}
    \]
\end{frame}

% Slide 7: Confidence Intervals - P-Hat (Population Proportion)
\begin{frame}{Computing \(\hat{p}\) (Proportion of Values \(\geq 4\))}

    \textbf{Example:} Given a set of samples, we compute the proportion of values in each sample that are at least 4 and store the result in a vector.

    \begin{block}{R Code}
# Generate example data: 10 samples, each with 20 random values (1 to 6)\\
set.seed(123)  \textbf{For reproducibility}\\
samples $=$ replicate(10, sample(1:6, 20, replace = TRUE))\\

 Compute p-hat for each sample (proportion of values $>=$ 4)\\
p\_hat $=$ apply(samples, 2, function(x) mean(x $>=$ 4))\\

 Display p\_hat values\\
print(p\_hat)
    \end{block}

\end{frame}





% Slide 8: Distribution Quantiles
\begin{frame}{Distribution Quantiles}
    \begin{itemize}
        \item Functions in R to obtain quantiles from different distributions:
        \begin{itemize}
            \item \texttt{qnorm()} - Normal distribution
            \item \texttt{qt()} - t-distribution
            \item \texttt{qchisq()} - Chi-squared distribution
        \end{itemize}
    \end{itemize}
\end{frame}

\begin{frame}{Distribution Quantiles}

    \begin{block}{R Code: Compute Quantiles Using a Loop Function}
\texttt{compute\_quantiles <- function(prob\_values, df\_value) \{}
\begin{itemize}
    \item Loop through each probability value \texttt{p}.
    \item Compute quantiles using:
    \begin{itemize}
        \item \texttt{qnorm(p)} - Normal distribution
        \item \texttt{qt(p, df)} - t-distribution
        \item \texttt{qchisq(p, df)} - Chi-squared distribution
    \end{itemize}
    \item Store and return results.
\end{itemize}
\texttt{\}}

\texttt{\# Example Usage:}  
\texttt{quantile\_results <- compute\_quantiles(c(0.025, 0.5, 0.975), 10)}  
\texttt{print(quantile\_results)}
    \end{block}

\end{frame}

% Slide 9: Lab Submissions
\begin{frame}{Lab Submissions}
    \begin{itemize}
        \item \textbf{R script:}
        \begin{itemize}
            \item Must be commented.
            \item If we run the script you turn in, it must run without errors and give the same output that you have in your report.
        \end{itemize}

        \item \textbf{Report:}
        \begin{itemize}
            \item Template available on Canvas.
            \item Some sections require you to add descriptions/summaries.
        \end{itemize}

        \item \textbf{CSV File:} Required for lab submissions.

        \item \textbf{Updated Note Sheet:} Must be submitted weekly, including notes from all previous labs.
    \end{itemize}
\end{frame}

\end{document}
