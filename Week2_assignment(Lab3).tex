

\documentclass{article}
\usepackage{geometry}
\geometry{a4paper, margin=1in}
\usepackage{hyperref}
\usepackage{fancyhdr}
\pagestyle{fancy}
\usepackage{enumitem}
\usepackage{graphicx}
\fancyhead[L]{STAT 3611 Assignment (Lab 3)}
\fancyhead[R]{Feb 10 at 11:59pm}
\setlist[itemize]{topsep=2pt, itemsep=1pt, left=10pt}
\setlist[enumerate]{topsep=2pt, itemsep=1pt, left=10pt}


\begin{document}



\section*{Topic Overview}
In this assignment you will practice the following skills in R:
\begin{itemize}
    \item Work with one of R’s built-in datasets.
    \item Computing summary statistics for a column of data.
    \item Isolating a subset of a data frame.
    \item Creating a new vector of data.
    \item Creating histograms.
    \item Creating boxplots.
    \item Using the sort function to find order statistics.
\end{itemize}

\section*{Submission Instructions}
For this lab, you will submit 3 files:
\begin{itemize}
    \item \textbf{Code File:} Must have a \texttt{.r} extension and be named \texttt{Lab3\_LastNameFirstName.r}. 
    \item \textbf{Lab Report:} Must have a \texttt{.pdf} extension and be named \texttt{Lab3\_LastNameFirstName.pdf}.
    \item \textbf{Reference Note Sheet:} Must have a \texttt{.pdf} extension and be named \texttt{Lab3\_Notes\_LastNameFirstName.pdf}. The reference sheet should contain all information from previous labs and any additional notes about new functions/code blocks/tips acquired in this lab.
\end{itemize}

\section*{Lab Instructions}
This lab will use the \texttt{quakes} dataset in R. The documentation can be found here:
\url{https://stat.ethz.ch/R-manual/R-devel/library/datasets/html/quakes.html}

\subsection*{Initial Data Overview}
\begin{itemize}
    \item Load the \texttt{quakes} dataset in R.
    \item What are the column headers for this dataset?
    \item How many rows of data are in the dataset?
\end{itemize}

\subsection*{Summary Stats for the Full Dataset}
Compute the following for the magnitudes of the recorded earthquake data:
\begin{itemize}
    \item Minimum value
    \item Maximum value
    \item Range
    \item First quartile
    \item Median
    \item Third quartile
    \item Interquartile range
    \item Mean
    \item Variance
    \item Standard deviation
    \item Coefficient of variation
\end{itemize}

\subsection*{Data Visualization for the Full Dataset}
All plots must have a descriptive title and appropriate axis labels:
\begin{itemize}
    \item Plot a histogram of the depths recorded for all earthquakes with the height of each bar labeled.
    \item Plot a histogram of the magnitudes recorded for all earthquakes with the height of each bar labeled.
    \item Create a boxplot for the number of stations that recorded data for each earthquake.
\end{itemize}

\subsection*{Manipulating Data and Creating New Vectors}
\begin{itemize}
    \item Create a new vector with the depth of each earthquake in miles (originally in kilometers).
    \item Create two new data frames:
    \begin{itemize}
        \item Earthquakes recorded by fewer than 40 stations.
        \item Earthquakes recorded by at least 40 stations.
    \end{itemize}
\end{itemize}

\subsection*{Summary Stats for Data Subsets}
\begin{itemize}
    \item How many earthquakes were recorded by fewer than 40 stations?
    \item How many earthquakes were recorded by at least 40 stations?
    \item Compute the 5-number summary for the depths:
    \begin{itemize}
        \item All depths in miles.
        \item Depths recorded by fewer than 40 stations in kilometers.
        \item Depths recorded by at least 40 stations in kilometers.
    \end{itemize}
    \item Compute the mean, variance, and interquartile range for:
    \begin{itemize}
        \item Depths recorded by fewer than 40 stations in kilometers.
        \item Depths recorded by at least 40 stations in kilometers.
    \end{itemize}
\end{itemize}

\subsection*{Data Visualization for Data Subsets}
All plots must have a descriptive title and appropriate axis labels. Change the plot settings so that two plots are presented side by side using the \texttt{par(mfrow=c(\#, \#))} command in R:
\begin{itemize}
    \item Plot histograms of magnitudes for earthquakes recorded by fewer than 40 stations (left) and at least 40 stations (right).
    \item Plot boxplots of depths for earthquakes recorded by fewer than 40 stations (left) and at least 40 stations (right). Use the \texttt{ylim} parameter to ensure both boxplots have the same y-axes. Explain your choice and the benefits of using the same axes for side-by-side plots.
\end{itemize}

\subsection*{Order Statistics}
\begin{itemize}
    \item Create a new vector with the longitude values for earthquakes recorded by at least 40 stations, ordered from smallest to largest.
    \item What is the 18th order statistic for the longitude values of earthquakes recorded by at least 40 stations?
    \item Write a function (your name) that takes two numbers (\texttt{x, y}) and returns \texttt{x * y}.
\end{itemize}

\end{document}
