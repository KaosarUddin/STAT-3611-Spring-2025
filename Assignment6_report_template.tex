\documentclass[a4paper,12pt]{article}
\usepackage{graphicx}
\usepackage{amsmath, amssymb}
\usepackage{booktabs}

\title{STAT 3611 – Lab 6 Report}
\author{[Your First and Last Name]}
\date{[Submission Date]}

\begin{document}

\maketitle

\section{Introduction}
This report presents the analysis conducted in Lab 6, focusing on statistical computations and sampling techniques using the \texttt{faithful} dataset in R. The objectives of this lab include data exploration, computing summary statistics, generating confidence intervals, and working with sampling techniques.

\section{Faithful Data Overview}
The \texttt{faithful} dataset in R contains [number] rows and [number] columns. The column headers are:



\section{Faithful Data Summaries}
Summary statistics for the two variables in the dataset are computed as follows:

\subsection{Eruption Duration}
\begin{itemize}
    \item Mean: [value]
    \item Population Variance: [value]
    \item Population Standard Deviation: [value]
    \item Population Coefficient of Variation: [value]
\end{itemize}

\subsection{Waiting Time}
\begin{itemize}
    \item Mean: [value]
    \item Population Variance: [value]
    \item Population Standard Deviation: [value]
    \item Population Coefficient of Variation: [value]
\end{itemize}

\section{Sample Analysis Aggregate Summary}
Using the vectors where sample mean, variance, and standard deviation were computed for 120 samples from the eruption duration column, the following summaries are obtained:

\begin{table}[h]
    \centering
    \begin{tabular}{lccc}
        \toprule
        & Sample Means & Sample Variance & Sample Standard Deviation \\
        \midrule
        Average & [value] & [value] & [value] \\
        Population Variance & [value] & [value] & [value] \\
        Bias & [value] & [value] & [value] \\
        \bottomrule
    \end{tabular}
    \caption{Summary Statistics for Sample Analysis}
\end{table}

\section{Quantiles}
\begin{itemize}
    \item The z-value for a 2-sided 95\% confidence interval is [value].
    \item The t-value for a 2-sided 95\% confidence interval with $n = 10$ is [value].
\end{itemize}

\section{Confidence Intervals}
For the 120 generated confidence intervals:

\begin{itemize}
    \item The true mean is within the generated confidence interval [number] out of 120 times.
    \item This is/is not what was expected because [explanation].
\end{itemize}

\section{Conclusion}
Summarize the key findings of this lab, discussing the accuracy of sample estimates, confidence intervals, and any observed biases. Reflect on the importance of sampling in statistical analysis.

\section{References}
List any references, such as R documentation, lecture notes, or textbooks used to complete this lab.

\end{document}
